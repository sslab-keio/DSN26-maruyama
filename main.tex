\documentclass[conference]{IEEEtran}
\IEEEoverridecommandlockouts
% The preceding line is only needed to identify funding in the first footnote. If that is unneeded, please comment it out.
\usepackage{cite}
\usepackage{amsmath,amssymb,amsfonts}
\usepackage{algorithmic}
\usepackage{graphicx}
\usepackage{textcomp}
\usepackage{xcolor}

% added
\usepackage{subcaption}
\usepackage{booktabs}
\usepackage{dblfloatfix}

\def\BibTeX{{\rm B\kern-.05em{\sc i\kern-.025em b}\kern-.08em
    T\kern-.1667em\lower.7ex\hbox{E}\kern-.125emX}}
\begin{document}

\newcommand{\sysname}{X-Monitor}

\title{{\sysname}: XDP-based Lightweight Monitoring \\ for Cloud-native Services}

% \author{\IEEEauthorblockN{1\textsuperscript{st} Given Name Surname}
% \IEEEauthorblockA{\textit{dept. name of organization (of Aff.)} \\
% \textit{name of organization (of Aff.)}\\
% City, Country \\
% email address or ORCID}
% \and
% \IEEEauthorblockN{2\textsuperscript{nd} Given Name Surname}
% \IEEEauthorblockA{\textit{dept. name of organization (of Aff.)} \\
% \textit{name of organization (of Aff.)}\\
% City, Country \\
% email address or ORCID}
% \and
% \IEEEauthorblockN{3\textsuperscript{rd} Given Name Surname}
% \IEEEauthorblockA{\textit{dept. name of organization (of Aff.)} \\
% \textit{name of organization (of Aff.)}\\
% City, Country \\
% email address or ORCID}
% \and
% \IEEEauthorblockN{4\textsuperscript{th} Given Name Surname}
% \IEEEauthorblockA{\textit{dept. name of organization (of Aff.)} \\
% \textit{name of organization (of Aff.)}\\
% City, Country \\
% email address or ORCID}
% \and
% \IEEEauthorblockN{5\textsuperscript{th} Given Name Surname}
% \IEEEauthorblockA{\textit{dept. name of organization (of Aff.)} \\
% \textit{name of organization (of Aff.)}\\
% City, Country \\
% email address or ORCID}
% \and
% \IEEEauthorblockN{6\textsuperscript{th} Given Name Surname}
% \IEEEauthorblockA{\textit{dept. name of organization (of Aff.)} \\
% \textit{name of organization (of Aff.)}\\
% City, Country \\
% email address or ORCID}
% }

\maketitle

\begin{abstract}
Cloud-native services have evolved from monolithic architectures to microservices, comprising hundreds to thousands of containerized components.  
These microservices are launched frequently and collaborate closely, and each must meet strict requirements for responsiveness and availability.
To satisfy service-level agreements (SLAs), monitoring plays a critical role in microservices. 
To enable an orchestration system to manage each microservice effectively, monitoring services must operate with both low latency and low overhead, allowing them to capture timely service status without compromising service performance.
However, conventional user-space monitoring tools such as Netdata and Prometheus incur scheduling delays and context switch overhead, resulting in increased monitoring latency and degraded service performance.

We present X-Monitor, a lightweight monitoring framework based on eXpress Data Path (XDP).  
X-Monitor has three key features:  
(1) In-kernel monitoring achieves low overhead by eliminating user-kernel context switches, as both packet processing and metric collection are performed entirely within the kernel.  
(2) SoftIRQ-level execution enables low-latency monitoring by reducing delay in executing these monitoring tasks without incurring scheduling latency and with minimal preemption.
(3) Programmable metric selection enables flexible in-kernel monitoring via user-defined XDP programs.
Our evaluation using Memcached and YCSB demonstrates that X-Monitor reduces the 99.9th percentile monitoring latency by two orders of magnitude and mitigates throughput degradation by up to 11.8\% compared to traditional monitoring systems.
\end{abstract}

\begin{IEEEkeywords}
% component, formatting, style, styling, insert
\end{IEEEkeywords}

\section{Introduction}
\label{sec:introduction}

\begin{figure*}[t]  % Use figure* for spanning across both columns
    \centering
    % First image (a)
    \hfill
    % \hfill
    \begin{minipage}[b]{0.36\textwidth}  % Adjust width as needed
        \centering
        \includegraphics[width=\textwidth]{fig/netdata-overview.pdf}
        \subcaption{Traditional Monitoring System}
        \label{fig:traditional-monitoring}
    \end{minipage}
    \hfill
    % Second image (b)
    \begin{minipage}[b]{0.36\textwidth}  % Adjust width as needed
        \centering
        \includegraphics[width=\textwidth]{fig/x-monitor-overview.pdf}
        \subcaption{{\sysname}}
        \label{fig:x-monitor}
    \end{minipage}
    \hfill
    \hfill
    \caption{Monitoring system overview. (a) In traditional monitoring systems, the metrics collector runs in user space, incurring scheduling delays and frequent context switches due to system calls. (b) In {\sysname}, the metrics collector runs in kernel space, avoiding such overhead.}
    \label{fig:monitoring-overview}
\end{figure*}

Cloud services have undergone a drastic shift from monolithic architectures to microservices.
Cloud-native applications are composed of tens or hundreds of microservices, each operating independently to deliver specific functionalities~\cite{qos-microservices, seer}.
These microservices typically run on containerized platforms, providing benefits such as scalability, isolation, and ease of deployment~\cite{xcontainers}.
Microservices require low latency and high availability to ensure they operate without failures or interruptions, and they are often deployed in high-density environments to maximize resource utilization.

Monitoring has become increasingly critical to ensure the service-level agreement (SLA) of cloud services, making the monitoring system an indispensable part of modern cloud operations~\cite{pivot, microview}.
Effective monitoring involves collecting both kernel and user metrics~\cite{GRANO} to coordinate tens or hundreds of microservices.
% This maintains the reliability of cloud-native environments.

Monitoring microservices presents unique requirements.
%
First, to ensure accurate decision-making, monitored metrics must be collected in real time.
In microservice-based architectures, the state of individual services changes frequently and unpredictably, while the continuous launch and termination of new services constantly change resource allocation.
For instance, AWS Lambda can start up to 15,000 containers per second for production workloads and is expected to scale further for heavier workloads~\cite{awscontainers}.
Failing to capture these frequent changes in a timely manner can lead to incorrect load balancing or false-positive failure detections, ultimately resulting in degraded system performance and reduced availability.
Therefore, low-latency monitoring is essential to maintain an up-to-date global view of the system and ensure SLA in such dynamic environments.

% Second, the monitoring system is desirable not to interfere with the performance of host services.
% As microservices are deployed in high-density environments, minor overheads caused by monitoring can have non-negligible degradation in service performance, potentially degrading user experience.
Second, the monitoring system should not interfere with the performance of host services.
In high-density environments, where microservices are densely packed and highly interdependent, even small monitoring overheads can lead to degradation in service throughput~\cite{zero, microview}, and also increase response latency and may cause request backlogs or timeouts, directly impacting user experience.
Moreover, performance degradation undermines the service’s ability to meet its SLA.
% In practice, on Alibaba Cloud during the W11 shopping festival, monitoring caused a 6.25\% reduction in Redis throughput due to the interference introduced by the monitoring process~\cite{zero}.
% To prevent this, it is crucial to design monitoring systems that minimize overhead and ensure that the performance of core services remains unaffected.

% However, traditional monitoring systems are often not sufficient to meet these requirements.
% Traditional monitoring systems, such as NetData~\cite{netdata} and Prometheus~\cite{prometheus}, are typically implemented as user-space processes.
% In high-density environments, where a large number of processes are actively running, the scheduling delays for allocating CPU resources to the monitoring process can become non-negligible. 
% These delays increase the monitoring latency, making it harder to collect real-time metrics effectively.
% Moreover, when the monitoring interval is reduced to capture finer-grained data, the frequent user/kernel context switches introduce considerable overhead.
% This overhead not only inflates the tail latency of monitoring messages but also leads to performance degradation in cloud services.
% As a result, the impact of monitoring on the performance of host services becomes more pronounced in such environments, undermining their reliability and efficiency.

Traditional monitoring systems are often not sufficient to meet these requirements.
Traditional monitoring systems, such as NetData~\cite{netdata} and Prometheus~\cite{prometheus}, are typically implemented as user-space processes.
In high-density environments, where a large number of processes are actively running, the scheduling delays for allocating CPU time to the monitoring process can become non-negligible. 
These delays increase the monitoring latency, making it harder to collect metrics in real time.
Moreover, since the monitoring interval must be short enough to collect metrics in real time, the frequent user/kernel context switches introduce considerable overhead.
This overhead not only increases the tail latency of monitoring messages but also leads to performance degradation in cloud services.
As a result, the negative side effects of monitoring on the performance of host services become more pronounced in such environments.
In fact, such performance issues have been observed in real-world deployments.
For example, during China's shopping festival ``double eleven'', Alibaba Cloud reported that the monitoring latency increased by more than 10$\times$, and the throughput of services decreased by 6.25\% due to the overhead introduced by monitoring~\cite{zero}.

We propose {\sysname}, a monitoring framework leveraging eXpress Data Path (XDP)~\cite{xdp}, to achieve low-latency and low-overhead monitoring.
XDP is a packet processing mechanism integrated into the Linux kernel, designed for handling packets directly at the network device driver level.
XDP allows users to write custom packet processing programs, which are compiled into eBPF bytecode, and then loaded into the kernel space.
The eBPF verifier~\cite{verifier} verifies the correctness of the program and ensures only safe programs are loaded and executed within the kernel space.
By leveraging XDP, X-Monitor works on conventional Ethernet NICs.
The key features of {\sysname} are the following.

\textbf{In-Kernel Monitoring. } 
In {\sysname}, both metric collection and packet processing are performed entirely within the kernel space.
This design eliminates user–kernel context switch overhead.

\textbf{SoftIRQ-Layer Execution. }
% By executing the monitoring logic at the SoftIRQ layer, the system reduces the delay between packet arrival and the start of monitoring.
% Furthermore, since SoftIRQ handlers run to completion and are only interrupted by HardIRQs, the monitoring logic executes without preemption, achieving consistently low latency.
By executing the monitoring logic at the SoftIRQ layer, the system reduces the delay between packet arrival and the start of monitoring.
Furthermore, since SoftIRQ handlers run to completion and are not preempted by anything other than HardIRQs, the monitoring logic executes with consistently low latency.

\textbf{Programmable Metric Selection. } 
% The proposed system allows users to flexibly select which metrics to collect based on their monitoring goals.
% Users can write custom monitoring logic as XDP programs, which are safely attached to the NIC after passing the eBPF verifier.
% In addition to selecting specific metrics, users can also apply in-kernel computation before transmitting the results to monitoring clients.
In {\sysname}, users write custom monitoring logic as XDP programs and load them into the kernel.
This approach enables fully in-kernel monitoring while preserving flexibility, as users can freely choose which metrics to collect and how to process them.
% Simple in-kernel computations can also be applied before sending metrics to the monitoring client.
In-kernel computations are possible to summarize the metrics before sending them to the monitoring client.

%% Before Revision
% However, there are challenges associated with obtaining metrics using existing XDP implementations.
% (1) The first challenge involves the verifier.
% Typically, programs designed to access metrics are rejected by the verifier, making it difficult to obtain the desired metrics.
% To address this issue, we extended the capabilities of eBPF by adding custom helper functions.
% These functions allow metric-accessing programs to pass the verifier’s safety checks while enabling secure access to the required metrics.
% (2) The second challenge relates to the requirement for non-blocking execution.
% Since XDP operates at the softIRQ layer, any metric collection performed in this context must be non-blocking.
% Blocking at the softIRQ layer can lead to delays in processing other critical system tasks, negatively impacting overall system performance.
% To ensure non-blocking behavior, we verified through code inspection that metric collection via \texttt{procfs} does not involve locking operations that could cause contention.
% We also proactively locked relevant user metric data in memory using \texttt{mlock}, ensuring that the user metrics would not be swapped out during execution and thus avoiding delays due to swap-in operations.
% (3) The third challenge is related to packet size limitations.
% Standard XDP is restricted to handling packets no larger than the MTU, which poses a problem for monitoring systems that need to transmit metrics exceeding the MTU size.
% To overcome this limitation, we leverage the XDP multi-buffer feature, introduced in Linux 5.18, which enables XDP to process packets that span multiple buffers~\cite{multibuffer}.

{\sysname} enables XDP programs to collect kernel metrics through \texttt{procfs}, an interface designed originally to expose kernel metrics to the user level.
By accessing the \texttt{procfs} interface from the inside of the kernel, {\sysname} can obtain all the kernel metrics \texttt{procfs} provides.
To collect user metrics, {\sysname} provides a shared memory space between the kernel and the user-space monitors.
{\sysname} enables XDP programs to access the shared memory to collect user metrics.

Implementing {\sysname} is not straightforward.
First, accessing kernel memory from an XDP program is prohibited to ensure memory safety.
Any program that attempts to access kernel memory is rejected by the verifier.
{\sysname} adds custom helper functions that allow XDP programs to access kernel/user metrics without being rejected by the eBPF verifier.
Second, non-blocking execution must be guaranteed in XDP programs.
Blocking at the SoftIRQ layer can cause delays in other critical system tasks, which may negatively affect overall system performance. 
To ensure that the added helper functions are non-blocking, it must be guaranteed that neither the helper functions nor any kernel functions they invoke perform blocking operations such as acquiring locks, and that they do not access memory pages that may be subject to swapping.
We confirmed through code inspection that metric collection via \texttt{procfs} does not involve any locking.
We also pinned the corresponding memory regions in physical memory to prevent user metrics from being swapped out.

We evaluate {\sysname} to show that it achieves lower monitoring latency and causes less interference with host service throughput compared to Netdata.
Our experimental results demonstrate that {\sysname} outperforms Netdata.
Specifically, it reduces the 99.9th-percentile monitoring latency by 2 orders of magnitude.
% In YCSB workloads using Memcached, it reduces the throughput degradation by 11.8\% compared to Netdata.  
In YCSB workloads using Memcached, Netdata caused an 11.8\% reduction in throughput, whereas {\sysname} showed no degradation.

The contributions of this paper can be summarized as follows:
\begin{itemize}
  \item We propose {\sysname}, a novel monitoring framework that leverages XDP to achieve low-latency and low-overhead monitoring in high-density, performance-critical cloud environments. {\sysname} executes entirely in-kernel to eliminate context switches and minimize monitoring delays.
  % \item We address key challenges in integrating metric collection into the XDP layer: enabling verifier-safe access to kernel and user-space metrics via custom eBPF helper functions, and achieving non-blocking execution at the softIRQ level by inspecting kernel interfaces and using \texttt{mlock} to pin user-space metric memory.
  \item We address key challenges in integrating metric collection into the XDP layer: safe access to kernel and user-space metrics within the constraints of the eBPF verifier, non-blocking execution within the SoftIRQ context.
  \item We evaluate {\sysname} on Memcached and YCSB workloads, demonstrating lower monitoring latency and reduced overhead compared to Netdata.
\end{itemize}

The rest of this paper is organized as follows:
Section \ref{sec:background} describes the background and motivation of our research. 
Section \ref{sec:proposal} describes the overview and implementation of our proposal.
Section \ref{sec:design-implementation} shows the system design of our framework. 
% Section \ref{sec:case-study} shows the case study of both kernel and user monitoring. 
Section \ref{sec:evaluation} shows the experimental results. 
Section \ref{sec:related-work} describes related works. 
Section \ref{sec:conclusion} concludes the paper.
\section{Background and Motivation}
\label{sec:background}

% \begin{figure}[t]
%     \centering
%     % First image (a)
%     \begin{minipage}[b]{0.48\linewidth}
%         \centering
%         \includegraphics[width=\linewidth]{SoCC-2025/figures/experiment/cdf-pre-ycsb-latest.png}
%         \subcaption{CDF of Monitoring Latency.}
%         \label{fig:pre-cdf}
%     \end{minipage}%
%     \hspace{0.02\linewidth}  % Small space between images
%     % Second image (b)
%     \begin{minipage}[b]{0.48\linewidth}
%         \centering
%         \includegraphics[width=\linewidth]{SoCC-2025/figures/experiment/memcached-throughput.png}
%         \subcaption{Memcached Throughput.}
%         \label{fig:pre-throughput}
%     \end{minipage}
%     \caption{Monitoring latency of Netdata and throughput of Memcached under low and high loads.}
%     \label{fig:pre-experiment}
% \end{figure}


%  \begin{figure*}[t]
%     \centering
%     \includegraphics[width=0.90\textwidth]{SoCC-2025/figures/illustration/monitoring-overview.pdf}
%     \caption{Monitoring system overview. (a) In traditional monitoring systems, the metrics collector runs in user space, incurring scheduling delays and frequent context switches due to system calls. (b) In {\sysname}, the metrics collector runs in-kernel, avoiding such overhead.}
%     \label{fig:monitoring-overview}
% \end{figure*}

% TODO: qos-microservice に,あるマイクロサービスの遅延が他のマイクロサービスに伝播していく,という話が書いてあるのでした方が良いか.
% 一つ一つのマイクロサービスのメンテナンスが重要という話につながる.


\subsection{Traditional Monitoring System}
\label{subsec:traditional-monitoring}

We begin by introducing traditional monitoring techniques and highlighting their limitations.
Conventional monitoring tools, such as Netdata~\cite{netdata} and Prometheus~\cite{prometheus}, are widely adopted in cloud environments due to their flexible metric collection capabilities.
By allowing users to select and collect metrics tailored to the characteristics and goals of each microservice, these tools enable fine-grained observability, which is a key advantage in modern cloud-native systems.

In these systems, the metrics collector typically runs as a user-space process and collects both user and kernel metrics (Fig.~\ref{fig:traditional-monitoring}).
For kernel metrics, the collector issues system calls to access \texttt{procfs}~\cite{procfs}, a pseudo file system that exposes various kernel metrics stored in memory.
For instance, by reading files such as \texttt{/proc/stat} and \texttt{/proc/meminfo}, one can obtain CPU and memory usage statistics.
% For user metrics, the collector queries dedicated APIs exposed by each microservice, which are typically implemented for the purpose of monitoring.
% Examples of user metrics include request rate, error rate, and response latency specific to each service.
% For user metrics, such as request rate, error rate, and response latency, are collected by querying dedicated APIs exposed by each microservice for monitoring purposes.
To collect user metrics, such as request rate, error rate, and response latency, the metrics collector queries APIs provided by each microservice.
It sends a request for user metrics to the microservice---for example, the \texttt{stats} command in Memcached or the \texttt{INFO} command in Redis.
Upon receiving the request, the microservice writes the metrics to the socket using the \texttt{write} system call, and the metrics collector retrieves them by reading from the socket using the \texttt{read} system call.

However, this conventional design struggles to satisfy the low-latency and low-overhead requirements of modern monitoring workloads, primarily for the following reasons:

\textbf{Monitoring delay due to resource contention. }  
Since the metrics collector is scheduled as a user-space process, its execution is subject to CPU scheduling delays.
In high-density cloud environments where CPU resources are tightly packed, collectors may face significant waiting time before being allocated CPU time.
% Similarly, if the target microservice is not currently scheduled on a CPU, the monitoring API call may be delayed, increasing response latency.
As illustrated in Fig.~\ref{fig:traditional-monitoring}, an incoming packet requesting the metrics goes through the network stack in the kernel and then is passed to the metrics collector running as a user process. This collector invokes some system calls to obtain kernel metrics through \texttt{procfs} or performs an inter-process communication (IPC) to get user metrics.
The network stack used in traditional monitoring can cause unexpected latency due to internal lock contention.
% While CPU binding~\cite{iron} can be used to reduce latency by pinning monitoring processes to specific cores, it may increase contention with co-located microservices and degrade overall performance.
While CPU binding~\cite{iron} can be used to reduce latency by pinning monitoring processes to specific cores, it reduces the number of cores available to co-located microservices, which may degrade overall performance.

\textbf{Monitoring overhead from frequent context switches. }  
User-level monitoring involves multiple context switches between the user and kernel space.
% (i) Switching from packet processing in the network stack to the user-space metrics collector triggers a context switch due to scheduling, (ii) system calls issued for kernel metric access, and (iii) system calls to retrieve user metrics---all introduce user/kernel context switches.
As shown in Fig.~\ref{fig:traditional-monitoring}, (i) switching from packet processing in the network stack to the user-space metrics collector triggers a context switch, (ii) issuing system calls to access kernel metrics involves a context switch, and (iii) invoking system calls to retrieve user metrics also results in a context switch.
% The number of system calls increases proportionally with the number of metrics and monitored microservices. %% deleted
% Additionally, the TCP/IP stack itself incurs significant overhead, often becoming a primary bottleneck in end-to-end latency and throughput~\cite{stackmap}. % deleted


%% memo: メトリクスサイズの説明をユーザがメトリクスを選択できるというにように組み込んだ説明
% The metrics collector selects and returns the required metrics to the client based on user requests.
The metrics collector selects the metrics, specified by the users, and returns them to the client.
% To make the large volume of collected metrics from \texttt{procfs} and microservice API easier for clients to interpret, the collector organizes them into logical groups called charts.
In the case of kernel metrics, Netdata sends approximately 280 bytes for CPU metrics, 180 bytes for memory metrics, and 900 bytes for disk metrics.
For user metrics, the metrics size is approximately 650 bytes for Memcached, 2030 bytes for Redis, and 300 bytes for Nginx.
Monitoring clients can selectively retrieve only the necessary metrics for their purposes. 
% For example, if the client wishes to retrieve the chart \texttt{redis\_local.connections}, which includes multiple metrics related to Redis connection counts, it can specify \texttt{chart=redis\_local.connections} in the HTTP request to obtain the chart~\cite{netdata-redis}.

Monitoring latency and overhead cause significant negative side effects on cloud-native services.
Alibaba Cloud reports that online shopping services experienced unacceptable degradation of throughput and increased tail latency on ``double eleven'' shopping festival.
Netdata was used to monitor a heavily loaded machine running Redis services with a 1s sampling interval.
The monitor process interferes with Redis services, degrading the throughput by 6.25\% and increasing the tail latency by 2$\times$ periodically.

% To confirm the negative side effects of user-level monitoring, we have conducted an experiment where Netdata is used as the monitoring tool and YCSB generates requests to a Memcached instance running on the same server.
To confirm the negative side effects of user-level monitoring, we conducted an experiment.
In this experiment, we used Netdata as the monitoring tool, and we employed YCSB to impose load on a Memcached instance by generating requests.
As shown in Fig.~\ref{fig:pre-cdf}, while the monitoring latency remains low under light load, the $99^{\mathrm{th}}$ percentile latency increases sharply under heavy load.
Fig.~\ref{fig:pre-throughput} shows that the presence of Netdata reduces YCSB throughput by 11.8\%.

\subsection{Monitoring in Cloud-native Services}
\label{subsec: cloud-monitoring}
% To ensure compliance with service-level agreement (SLA) in cloud environments, monitoring plays an indispensable role. 
% Effective monitoring involves the collection of both user metrics and kernel metrics to provide comprehensive insights into service performance and behavior.
% User metrics, such as CPU utilization and memory occupation, offer critical information about resource usage and overall system health, enabling administrators to identify potential bottlenecks and optimize performance~\cite{monitoring-survey}.
% Furthermore, cloud providers can improve resource utilization by analyzing the historical state of kernel metrics~\cite{resource-central, hcloud, scavenger}.
% On the other hand, user metrics, such as request latency, throughput, and error rates, provide valuable insights into the behavior and efficiency of individual services.
% These metrics allow for the detection of application-specific issues and support the maintenance of reliable and scalable operations in dynamic cloud environments.

% Monitoring plays an indispensable role in cloud-native services to comply with service-level agreements (SLAs).
% Monitoring services need to collect both kernel and user metrics to provide comprehensive insights into service performance and behavior.
% Kernel metrics, such as CPU utilization and memory consumption, provide critical information about resource usage and overall system health, enabling administrators to identify potential bottlenecks and optimize performance~\cite{monitoring-survey}.
% Furthermore, cloud providers can improve resource utilization by analyzing the historical state of kernel metrics~\cite{resource-central, hcloud, scavenger}.

Monitoring plays an indispensable role in cloud-native services to comply with service-level agreements (SLAs).
Monitoring services need to collect both kernel and user metrics to provide comprehensive insights into service performance and behavior.
Kernel metrics, such as CPU utilization, memory pressure, and I/O wait time, provide concrete indicators of how system resources are being consumed.
If sustained high CPU usage is observed across multiple cores, it may indicate contention among co-located containers. 
% Similarly, if frequent memory swapping occurs, it can signal that the node is under memory pressure, which may require adjustments to container placement or resource limits.
These observations help administrators detect performance bottlenecks, investigate abnormal behavior, and fine-tune resource allocation policies~\cite{monitoring-survey}.
% Moreover, by analyzing historical trends in kernel metrics, cloud operators can improve resource utilization through actions such as consolidating underutilized containers onto fewer nodes, balancing workloads more evenly across the cluster, or adjusting resource quotas to better reflect actual usage patterns~\cite{resource-central, hcloud, scavenger}. 
Moreover, cloud operators can improve resource utilization by analyzing historical trends in kernel metrics.
Based on these insights, they may consolidate underutilized containers, balance workloads across the cluster, or adjust resource quotas to match actual usage patterns~\cite{resource-central, hcloud, scavenger}.
% Moreover, cloud operators can improve resource utilization by analyzing historical trends in kernel metrics.
% For example, they can consolidate underutilized containers onto fewer nodes.
% They can also balance workloads more evenly across the cluster or adjust resource quotas to reflect actual usage patterns~\cite{resource-central, hcloud, scavenger}.
User metrics, such as request latency, throughput, and error rates, offer essential insights into how individual services behave under real-world workloads.
% These metrics are particularly valuable because they capture the internal state and operational quality of each microservice, which cannot be inferred solely from system-level observations.
They are especially useful for capturing aspects of service performance that are not visible at the system level.
% For instance, Redis exposes a metric called \texttt{redis.connections}, which counts client connection attempts.
% Monitoring this metric enables alerts such as \texttt{redis\_connections\_rejected}, which signals when Redis rejects new connections due to reaching the \texttt{maxclients} limit.
For example, Redis exposes \texttt{redis.connections}, which reveals connection load and helps detect rejections due to the \texttt{maxclients} limit, enabling timely configuration adjustments.
% If such rejections are detected, operators can promptly identify capacity saturation, investigate client behavior, and adjust configuration parameters such as the maximum connection limit or connection pooling strategies.
% By analyzing trends in these user metrics, cloud operators can detect early signs of service degradation, validate the effectiveness of configuration changes, and make informed decisions about autoscaling, request throttling, or architecture-level optimizations.

In microservices architectures, monitoring is conducted for each individual microservice.
This granular monitoring inherently incurs higher costs due to the increased number of components that need to be tracked.
Furthermore, compared to monolithic monitoring, microservices require more frequent monitoring to capture the rapid changes and interactions between services that are characteristic of these architectures~\cite{microview}. 
These characteristics of cloud-native services have introduced the need to address two new challenges in monitoring.

\textbf{Monitoring messages must be collected in a timely fashion. }
% Leading cloud-native services like Netflix~\cite{netflix} and Uber~\cite{uber} deploy hundreds or thousands of microservices, each requiring detailed and continuous monitoring to ensure system availability and performance.
% These microservices exhibit highly dynamic behavior~\cite{unified-monitoring}, with key performance indicators such as request latency and error rate fluctuating over timescales as short as seconds or even milliseconds~\cite{zero}.
% For example, AWS Lambda can launch up to 15,000 containers per second to handle production workloads, and this scale is expected to increase further~\cite{awscontainers}.
% Such rapid elasticity means that even a brief delay in monitoring, such as a few hundred milliseconds, can result in missing transient performance anomalies.
% These include sudden surges in request failures, latency spikes in user-facing services, or load imbalance across containers.
% If undetected, such issues can quickly cascade, leading to SLA violations, degraded user experience, or service outages.
% Therefore, to maintain SLA in such volatile environments, monitoring systems must capture and analyze metrics with millisecond-level granularity and minimal latency.
To maintain SLA in cloud-native services, monitoring systems must capture and analyze metrics with minimal latency.
Even a brief delay in monitoring, such as a few hundred milliseconds, can result in missing transient performance anomalies, including sudden surges in request failures, latency spikes in user-facing services, or load imbalance across containers.
If these issues go undetected, they can cascade and lead to SLA violations, degraded user experience, or service outages~\cite{qos-microservices}.
This requirement is especially critical in modern cloud platforms.
For example, AWS Lambda can launch up to 15,000 containers per second to handle production workloads, and this scale is expected to increase further~\cite{awscontainers}.
Microservices in large-scale platforms like Netflix~\cite{netflix} and Uber~\cite{uber} exhibit highly dynamic behavior~\cite{unified-monitoring}, with key performance indicators such as request latency and error rate fluctuating over timescales as short as seconds or even milliseconds~\cite{zero}.
Each of these microservices requires detailed and continuous monitoring to ensure system availability and performance.



\textbf{Monitoring must not interfere with cloud services. }
To ensure the performance and availability of cloud-native services, it is critical that monitoring does not interfere with the operation of microservices.
Modern deployments often colocate a large number of microservices on a single physical host to maximize resource utilization~\cite{resource-central}.
In such high-density environments, even lightweight monitoring tasks can compete for scarce CPU and memory resources.
This resource contention can lead to increased latency or missed deadlines in latency-sensitive services, which may violate the SLA.
Therefore, minimizing the resource consumption of monitoring systems is essential to maintaining service quality.

\begin{figure}[t]
    \centering
    \includegraphics[width=0.7\columnwidth]{fig/xdp-overview.pdf}
    \caption{
    XDP programs are loaded into the device driver layer after the eBPF verifier verifies their safety. 
    When a packet arrives, the XDP program is executed by the eBPF virtual machine. 
    Depending on the program’s logic, the packet can be: 
    (1) forwarded to the network stack in the kernel (\texttt{XDP\_PASS}), 
    (2) sent back immediately (\texttt{XDP\_TX}), or 
    (3) dropped (\texttt{XDP\_DROP}).
    }
    \label{fig:overview-xdp}
\end{figure}
\section{X-Monitor}
\label{sec:proposal}

% \begin{figure*}[t]
%    \centering
%     \includegraphics[width=0.75\textwidth]{SoCC-2025/figures/illustration/run-to-completion.pdf}
%    \caption{Execution Model Comparison. (1) In traditional monitoring systems, execution may be delayed due to lock contention within the network stack and waiting for CPU scheduling. (2) In contrast, {\sysname} adopts a run-to-completion model: once execution begins, it proceeds without being preempted or blocked by lock contention or scheduling delays, ensuring uninterrupted processing from start to finish.
%    }
%    \label{fig:run-to-completion}
%\end{figure*}

We propose {\sysname}, a lightweight monitoring system for cloud-native services that achieves low-latency monitoring while minimizing interference with the execution of microservices.
{\sysname} leverages eXpress Data Path (XDP), a lightweight packet processing mechanism, and provides the following three key features:

\begin{enumerate}
    \item \textbf{In-Kernel Monitoring:} Monitoring is completed entirely in kernel space, eliminating user–kernel context switches.
    \item \textbf{SoftIRQ-Layer Execution:} Since monitoring is performed at the SoftIRQ layer, there is no need to wait for CPU allocation due to scheduling.
    \item \textbf{Programmable Metric Selection:} Since the monitoring logic is implemented as an XDP program, users can flexibly choose which metrics to collect.
\end{enumerate}

Major cloud providers, including Amazon Web Services (AWS), Microsoft Azure, and Google Cloud Platform, offer Ethernet-based NICs as the default option for most instance types.
% While advanced networking technologies such as Remote Direct Memory Access (RDMA) are available in these environments, they often involve additional costs or configuration overhead.
While advanced networking technologies such as Remote Direct Memory Access (RDMA) are available in these environments, their usage is limited to specific instance types, which are often expensive.
For example, AWS offers Elastic Fabric Adapter (EFA), an RDMA-capable NIC designed for HPC and ML workloads.
EFA is only available on a limited set of instance types, such as c8gn.48xlarge, m6i.metal, and p6-b200.48xlarge~\cite{aws-efa-doc}.
These EFA-enabled instances are generally more expensive, making RDMA less accessible for cost-sensitive deployments.


To demonstrate that {\sysname} can support diverse and practical monitoring use cases, we examine how typical metrics are managed in most software systems.

\textbf{Metric Characteristics. }
Most monitoring metrics are simple counters that are updated in place after initialization.
Since the memory addresses of these counters remain unchanged,
{\sysname} can reliably collect metrics by referencing the memory addresses registered during initialization.
These characteristics are widely observed across both kernel and user metrics---such as per-CPU packet counts, memory usage statistics, and request rates.
As a result, {\sysname} can support a broad range of metrics comparable to traditional monitoring systems, while maintaining a lightweight and efficient design.

\textbf{XDP Overview. }
{\sysname} is implemented using XDP, an eBPF-based framework that enables high-performance packet processing in the kernel space.
As illustrated in Fig.~\ref{fig:overview-xdp}, XDP programs are triggered upon packet arrival and execute at the device driver layer, specifically in the SoftIRQ context, immediately after the hardware interrupt.
This design allows XDP to bypass the conventional Linux networking stack—prior to \texttt{sk\_buff} allocation—thereby achieving low-latency packet processing.
To ensure the safety and reliability of kernel execution, XDP includes a static verification mechanism, known as verifier~\cite{verifier}, which enforces strict constraints on loaded programs.
Verifier checks for invalid memory accesses, unbounded loops, and other unsafe behaviors, allowing only safe and well-formed programs to be loaded into the kernel.

\textbf{Challenges. }
Building {\sysname} atop XDP introduces two main challenges.  
First, the monitoring logic must pass verifier’s checks, which is non-trivial when accessing memory regions used to store metrics.  
Direct access to external memory is typically disallowed by verifier, requiring a mechanism that safely exposes metric memory to the XDP context while satisfying all verification rules.

Second, {\sysname} must ensure that metric collection is strictly non-blocking.
Since XDP runs in the SoftIRQ context, any blocking operations—such as waiting for locks or accessing swapped-out memory—could stall the system and degrade performance.  
To ensure robust system behavior, {\sysname} guarantees that all metric accesses are safe and non-blocking during monitoring execution.
\section{Design and Implementation}
\label{sec:design-implementation}

In this section, we introduce our proposed monitoring framework, {\sysname}.
Section~\ref{subsec:overview-of-xmonitor} provides an overview of {\sysname}, Section~\ref{subsec:metrics-collection} explains the interface for metric collection, 
Section~\ref{subsec:non-blocking} describes the non-blocking monitoring, and Section~\ref{subsec:large-packet} illustrates how to process a large-sized packet within the XDP context.




\subsection{Overview}
\label{subsec:overview-of-xmonitor}
% Upon receiving a monitoring packet from the monitoring client, X-Monitor completes the metric collection entirely within the kernel space and sends a response back to the monitoring client.
% This means that X-Monitor operates in a run-to-completion manner, ensuring that the monitoring process runs from start to finish without being preempted, except for hardware IRQs.
% Therefore, {\sysname} performs low-latency, low-overhead monitoring by avoiding user-kernel context switches and scheduling delays, and by executing monitoring tasks at the softIRQ layer in a run-to-completion manner.
% Additionally, the ability to dynamically load XDP programs allows users to flexibly modify which metrics to collect.

% Upon receiving a monitoring packet, the metrics collector—implemented as an XDP program running on the eBPF virtual machine—collects both user-space and kernel-space metrics and sends them back to the monitoring client.

Fig.~\ref{fig:x-monitor} shows the overview of {\sysname}.
Upon receiving a packet, the Metrics Collector—an XDP program running on the eBPF virtual machine—is executed at an early stage, before the packet enters the network stack.
It first parses the packet header to determine whether the incoming packet is a monitoring packet.
If it is not, the program can invoke actions such as \texttt{XDP\_PASS} to forward the packet to the regular kernel processing pipeline, or \texttt{XDP\_DROP} to discard the packet.
If it is a monitoring packet, the collector retrieves the required metrics---both user-space and kernel-space---from within the kernel and sends them back to the monitoring client using \texttt{XDP\_TX}.

This monitoring approach achieves low latency, low overhead, and flexible metric selection for the following reasons.

\textbf{In-Kernel Monitoring. }
Monitoring with this approach incurs minimal overhead because both packet processing and metric collection are completed entirely within the kernel space, eliminating the need for user–kernel context switches.
In contrast, traditional monitoring systems incur multiple context switches—for example, when invoking the monitoring process or issuing system calls to retrieve kernel or user metrics—each introducing additional overhead.
By avoiding these context switches, our method reduces monitoring overhead.

\textbf{SoftIRQ-Layer Execution. }
Since monitoring is performed at the SoftIRQ layer, {\sysname} follows a run-to-completion execution model. 
In user-space approaches, as shown in the figure, delays can occur during both packet processing and metric collection due to contention for network stack locks or waiting for CPU scheduling. During metric collection, there is also a possibility that the CPU may be preempted by other processes. These delays become particularly significant in cloud environments where microservices are densely deployed.
In contrast, {\sysname} avoids such delays regardless of server load: there is no lock contention, no CPU scheduling delay, and no interruption except for HardIRQs. As a result, it achieves low-latency monitoring by minimizing both the time from packet arrival to processing and the time from packet processing to transmission.

\textbf{Programmable Metric Selection. }
In the proposed monitoring system, users can write custom monitoring logic as XDP programs and load them into the kernel.
This design enables in-kernel monitoring without sacrificing flexibility, as users have full control over which metrics to collect and how to collect them.
By defining their own programs, users can flexibly select the necessary metrics based on their specific monitoring goals.
Moreover, simple in-kernel computations can be performed on the collected metrics before sending them to the monitoring client, allowing for more efficient data handling when needed such as calculating the average CPU usage or detecting whether the memory usage exceeds a threshold.

\subsection{Interface for Metric Collection}
\label{subsec:metrics-collection}
To collect metrics from an XDP program, a key challenge arises: metric collection involves memory access that falls outside the bounds permitted by eBPF, which causes the program to fail verifier’s safety checks. 
To address this issue, we add a BPF helper function specifically designed for metric collection.
BPF helper functions are predefined interfaces in the Linux kernel that allow eBPF programs to interact with kernel objects or exchange data with the kernel.
While the verifier checks how a BPF helper function is called --- for example, ensuring type safety and calling context --- it does not analyze the internal behavior of the helper itself.
% This property allows us to safely extend the monitoring capabilities of XDP without compromising verifier security.
This property allows us to safely extend the monitoring capabilities of XDP without modifying the verifier, thereby preserving the kernel’s guarantees of memory safety and program isolation.
By introducing a dedicated BPF helper function for metrics access into the Linux kernel, users can collect necessary metrics in their monitoring programs while still passing verifier’s checks.
This approach enables flexible and safe in-kernel monitoring within the constraints of the eBPF execution model.

% To collect kernel metrics, the BPF helper function internally leverages \texttt{procfs}, a pseudo file system widely used by traditional monitoring systems to expose runtime kernel information.
% Unlike ordinary files backed by physical storage, \texttt{procfs} files are dynamically generated at access time by invoking kernel functions that read in-memory data structures.
% These files are typically accessed via the \texttt{read} system call from user space, but do not involve disk I/O; instead, they return data directly from kernel memory.

% Most kernel metrics provided through \texttt{procfs} are maintained as counters that are continuously updated by the kernel to reflect the current system state.
% These include statistics such as CPU usage (\texttt{/proc/stat}), memory utilization (\texttt{/proc/meminfo}), process information (\texttt{/proc/\textit{pid}}), and network activity (\texttt{/proc/net/dev}).
% Because the values are generated on demand, they offer accurate, real-time visibility into the operating system.

% The BPF helper function replicates this behavior by invoking the same internal kernel functions used by \texttt{procfs}.
% As a result, it enables eBPF programs to access a wide range of kernel metrics entirely within the kernel context, without relying on user-space transitions.
% This ensures that in-kernel monitoring retains the same observability as conventional monitoring tools.

% This approach is also safe.
% Since the helper relies on well-defined, read-only kernel interfaces already used in production by \texttt{procfs}, it avoids unsafe memory operations.
% Furthermore, the eBPF verifier statically analyzes the helper’s usage to guarantee memory safety and compliance with kernel constraints.
% By building on trusted kernel mechanisms and verifier-enforced safety checks, the helper enables robust and secure metric access.

% Crucially, because traditional monitoring systems depend on \texttt{procfs} for kernel metric collection, the proposed method achieves equivalent coverage.
% By accessing the same underlying data through verified in-kernel logic, our approach maintains compatibility with existing monitoring practices while avoiding the overhead of context switches and system calls.



\subsubsection{Collecting Kernel Metrics}

\begin{table}[t]
  \centering
  % \small
  \scriptsize
  \caption{Kernel functions used to obtain metrics.}
  \begin{tabular}{@{}lll@{}}
    \toprule
    Category & \texttt{/proc} File & \texttt{procfs} Internal Functions \\ \midrule
    CPU      & \texttt{/proc/stat}         & \texttt{kcpustat\_cpu\_fetch()} \\
    Disk     & \texttt{/proc/diskstats}    & \texttt{part\_stat\_read\_all()} \\
    Memory   & \texttt{/proc/meminfo}      & 
    \begin{tabular}[t]{@{}l@{}}
      \texttt{si\_meminfo()} \\
      \texttt{si\_swapinfo()} \\
      \texttt{vm\_memory\_committed()} \\
      \texttt{global\_node\_page\_state()} \\
      \texttt{total\_swapcache\_pages()} \\
      \texttt{si\_mem\_available()} \\
      \texttt{global\_node\_page\_state\_pages()} \\
      \texttt{global\_zone\_page\_state()}
    \end{tabular} \\
    Network  & \texttt{/proc/net/snmp}     & 
    \begin{tabular}[t]{@{}l@{}}
      \texttt{snmp\_get\_cpu\_field\_batch()} \\
      \texttt{snmp\_get\_cpu\_field64\_batch()}
    \end{tabular} \\
    \bottomrule
  \end{tabular}
  \label{tab:procfs}
\end{table}


To collect kernel metrics, the BPF helper function utilizes \texttt{procfs}, a pseudo file system commonly employed by traditional monitoring systems to expose runtime kernel information. By relying on the same interface as conventional user-space monitoring tools, the helper enables access to a wide range of kernel metrics, just as traditional monitoring processes do.

Files under \texttt{procfs} --- such as \texttt{/proc/stat}, \texttt{/proc/meminfo} --- are not backed by persistent storage, but are dynamically generated by kernel functions that read in-memory data structures.
These files expose metrics such as CPU usage, memory utilization, and network activity, most of which are maintained as continuously updated counters.
Because our BPF helper invokes the same internal functions used by procfs, it can access this rich set of metrics entirely within the kernel context, providing real-time visibility into kernel state as shown in Tab.~\ref{tab:procfs}.

This approach offers two key benefits.
First, it achieves the same level of observability as traditional user-space monitoring tools, while eliminating the overhead of context switches.
% Second, it is safe.
% The helper relies on well-defined, read-only kernel interfaces already used in production by procfs, avoiding unsafe memory access.
Second, memory safety is guaranteed because the helper simply invokes existing kernel-provided functions through well-defined interfaces already used in production by \texttt{procfs}. 
% Additionally, eBPF verifier statically checks helper usage to ensure memory safety and compliance with kernel constraints.
Additionally, memory access violations are avoided because the helper functions only invoke read-only kernel interfaces.
As long as the arguments are valid, these functions do not perform unsafe memory accesses.
eBPF verifier statically checks the types and value ranges of the arguments to ensure their correctness.
% The absence of blocking behavior is discussed in Sec.~\ref {subsec:non-blocking}.
The absence of blocking behavior is discussed in Sec.~\ref {subsec:non-blocking}.

\subsubsection{Collecting User Metrics}

\begin{figure}[t]
    \centering
    \includegraphics[width=0.7\columnwidth]{fig/metrics-registration.pdf}
    \caption{User Metrics Registration. Microservices register the metadata of metrics to metrics descriptor.}
    \label{fig:metrics-registration}
\end{figure}
% Our design enables safe and efficient retrieval of user metrics from various microservices in a lightweight and non-intrusive manner.
% As described in Sec.~\ref{sec:proposal}, user metrics are stored in a shared memory region between the kernel and the microservices, and this design allows the XDP program to directly access them with minimal latency.  
% Importantly, helper functions are designed to be general-purpose so that they can monitor a wide range of microservices.

To efficiently retrieve user metrics from various microservices, it is crucial to support lightweight access to a diverse range of metrics.
As detailed in Sec.~\ref{sec:proposal}, to enable low-latency retrieval of user metrics from XDP programs, these metrics are placed in shared memory that is accessible from the kernel.
Our design avoids invoking the APIs of the microservices to prevent interference.

User metrics are defined by each microservice and are typically organized into multiple C structs (or equivalent representations) that store a set of performance-related fields.
For instance, Memcached utilizes seven metric structures, including \texttt{struct settings} and \texttt{struct rusage}.

Our system allows XDP programs to select which metrics to collect, aligning with existing monitoring tools.
Memcached and Redis provide the `stats` and `INFO` commands, respectively, which enable monitoring tools to specify the metrics to be collected.
Our design preserves this programmability while ensuring low latency and minimal overhead.

To enable XDP programs to retrieve user metrics, two preparatory steps are necessary. 
%
First, each user process must register its metrics with the kernel (See Fig.~\ref{fig:metrics-registration}).
This is done through a custom system call called \texttt{xmon\_reg\_metrics}, which allows the kernel to locate where the metrics are stored (the starting virtual address and the size of the metrics).
This call is invoked multiple times to register all structures related to the metrics.
The registered information is stored in a \emph{metrics descriptor} array allocated for each user process.
A \emph{metric descriptor}, which is an index into the descriptor array, is returned to the microservice.
When registering, the starting virtual address of the metrics is converted into a physical address.
The physical pages containing the metrics remain constant during execution, as discussed in Sec.~\ref{subsec:non-blocking}.
Since the kernel can access any physical address, the helper function can retrieve the registered metrics without needing address translation.

Second, after registering all relevant metric information, the microservice registers a port number that allows a monitoring client to access the user metrics. This port number is associated with the metrics descriptor array; thus, given the port number, an XDP program can look up the corresponding array and collect the user metrics from the registered addresses.

{\sysname}'s helper function retrieves user metrics and copies them into a network packet.
To allow an XDP program to select specific user metrics, it requires an array of metric descriptors as arguments.
The user metrics stored in the specified descriptors are copied into the packet, while the metrics not specified are omitted.
This helper function operates in a lock-free manner because it performs read-only operations on the user metrics and the descriptor array (see Sec.~\ref{subsec:non-blocking}).
Importantly, this design remains effective even if kernel space is separated from user space as a mitigation against Meltdown, since user metrics are accessed through their physical addresses.

One important consideration is that if a user metric registered in the metrics descriptor array exceeds 4 KB in size, it may span multiple physical memory pages.
In such cases, the helper might unintentionally read data from a different, unrelated kernel memory page.
To prevent this issue, the user metrics should be split into smaller components before being registered in the metrics descriptor array.

% Among the fields in these structs, only a subset is actually exposed to clients.  
% For instance, the \texttt{stats} command of Memcached (see Sec.~\ref{subsec:traditional-monitoring}) returns only 4 out of 70 fields in the \texttt{struct settings.

% To enable selective access to user metrics from XDP programs, we perform two preparatory steps.
% First, the fields required for monitoring are relocated to the beginning of the corresponding struct. 
% This allows the XDP program to efficiently copy only the relevant fields.  
% The selection of required metrics is aligned with those exposed by existing monitoring interfaces, such as Memcached's \texttt{stats} command and Redis's \texttt{INFO} command, to obtain the metrics provided by traditional user-space monitoring tools.
% Second, the microservice registers metadata about each metric struct in the kernel prior to monitoring.  
% This is done using a custom system call, \texttt{register\_metrics()}, which stores the metadata in a global \texttt{metrics array} (Fig.~\ref{fig:metrics-registration}).  
% This array is shared across all microservices in the system and contains the following fields for each entry:

% \begin{enumerate}
%   \item \textbf{\texttt{port\_number}}: Identifies the microservice.
%   \item \textbf{\texttt{id\_struct}}: Differentiates between multiple metric structs used by the same microservice.  
%         For example, Memcached has seven distinct structs, each with a unique ID.
%   \item \textbf{\texttt{size\_struct}}: The total size of the front portion of the struct that contains only the required metric fields.
%   \item \textbf{\texttt{PA\_struct}}: The physical address of the beginning of the struct.
% \end{enumerate}
% Since the physical page storing the metrics does not change during execution (see Sec.~\ref{subsec:non-blocking}), registration only needs to be performed once before monitoring begins.

% The helper function used by the XDP program is designed to access user metrics in a general and microservice-agnostic manner by relying solely on the information stored in the \texttt{metrics array}.  
% It takes \texttt{port\_number} and \texttt{id\_struct} as arguments, searches the \texttt{metrics array} for the matching entry, retrieves \texttt{size\_struct} and \texttt{PA\_struct}, and converts the physical address to a virtual address using the \texttt{\_\_va} macro.  
% It then copies the memory region from the starting address up to \texttt{size\_struct} bytes into the eBPF stack (Fig.~\ref{fig:memory-share}).

\subsection{Non-Blocking Monitoring}
\label{subsec:non-blocking}
Since XDP programs execute in the SoftIRQ context, they must not cause any blocking.
Blocking operations in such interrupt contexts can delay the handling of other interrupts or kernel tasks, potentially degrading overall system responsiveness.
To ensure safe and efficient monitoring, it is therefore crucial that metric collection remains non-blocking—specifically, it must avoid lock acquisition and prevent access to memory regions that may have been swapped out.
This design principle helps guarantee the safe execution of monitoring logic within the kernel.

In the context of kernel monitoring, to investigate whether any locking occurs in \texttt{procfs}, we analyzed the internal functions listed in Tab.~\ref{tab:procfs}, which are used to retrieve kernel metrics.
Our investigation confirmed that no lock acquisition is performed during metric retrieval.
In addition, it is known that kernel metrics are never swapped out ~\cite{zero}.
Therefore, metric collection via \texttt{procfs} can be performed without blocking, ensuring safe monitoring within the kernel.

% For user metrics, the systyem is designed to avoid acquiring locks during both metadata access and metric reading.
For user metrics, {\sysname} does not acquire locks during metadata access or metric reading because both operations are read-only and do not require synchronization.
However, user metrics may be subject to swap-out if no specific configuration is applied.
% To prevent blocking due to swapping, microservices are configured to pin the relevant memory using the \texttt{mlock} system call prior to monitoring.
To prevent blocking due to swapping, {\sysname} pins the relevant memory using the \texttt{mlock} system call before monitoring begins.
This ensures that user metric collection can also be performed safely without incurring any blocking behavior.

\subsection{Handling Large Size Packet}
\label{subsec:large-packet}
% BPF の最大スタックサイズの拡張および,XDP multi buffer の話をする.multibuffer をどう使えるかも書く

When using XDP for monitoring, one key limitation is the maximum packet size that XDP can handle.
XDP is not originally designed to process large packets, and therefore imposes restrictions on the size of packets it can operate on.
However, monitoring often involves collecting a large number of metrics from each microservice, which requires processing larger packets.

% In particular, there are two main constraints: the maximum size of transmittable packets and the limited size of the BPF stack that can be used within the program.

% \begin{figure}[htbp]
%     \centering
%     \includegraphics[width=0.7\columnwidth]{SoCC-2025/figures/illustration/xdp-multi-buffer.pdf}
%     \caption{XDP multi-buffer. The first buffer can be accessed through the \texttt{data} and \texttt{data\_end} members of the \texttt{xdp\_buff} structure. With multi-buffer support, additional buffers beyond the first one can be accessed via \texttt{skb\_shared\_info}, enabling the handling of multiple buffers.}
%     \label{fig:xdp-multi-buffer}
% \end{figure}

\textbf{XDP multi buffer support. }
XDP does not natively support the transmission or reception of packets that exceed the Maximum Transmission Unit (MTU).
Typically, the MTU is set to 1500 bytes, and even with Jumbo Frames, it is extended only up to 9000 bytes~\cite{jumbo-frame}.
Since monitoring involves collecting metrics from multiple microservice instances on a server simultaneously, this packet size is often insufficient.
For instance, each Memcached instance returns 656 bytes of metrics.


To overcome this limitation, the support for XDP multi-buffer~\cite{multibuffer, multibuffer-2}, introduced in Linux 5.18, enables the processing of packets larger than the MTU.
Conventional XDP programs assume a single buffer per packet and access packet contents using the \texttt{.data} and \texttt{.data\_end} pointers, which indicate the start and end addresses of the packet data.
eBPF verifier ensures memory safety by checking that all accesses fall within this range.
As a result, it was not possible to access data beyond a single buffer.
XDP multi-buffer addresses this limitation by allowing access to additional buffers through the \texttt{skb\_shared\_info} structure.
This capability makes it possible to process sufficiently large packets required for monitoring purposes.
By leveraging XDP multi-buffer, monitoring systems can handle larger packets without being constrained by traditional MTU limits, thereby expanding the scope of in-kernel packet processing.

% The BPF stack has a limited size, which restricts the amount of local memory available to eBPF programs.
% In Linux, the maximum BPF stack size is currently limited to 512 bytes, as defined by the macro \texttt{MAX\_BPF\_STACK}.
% However, this size is often insufficient for practical monitoring use cases.
% For instance, in the case of Memcached, existing monitoring tools collect metrics that require 656 bytes per instance—already exceeding the default stack limit.
% Moreover, as the number of monitored microservices increases, the memory required to store their corresponding metrics grows proportionally, further stressing the BPF stack usage.

% To address this limitation, we increased the value of \texttt{MAX\_BPF\_STACK} in the Linux kernel, thereby enabling our monitoring system to allocate a larger BPF stack size sufficient to store the necessary metrics.

% \textbf{BPF Stack size extension. }
% The BPF stack has a limited size, which restricts the amount of local memory available to eBPF programs.
% In Linux, the maximum BPF stack size is currently limited to 512 bytes, as defined by the macro \texttt{MAX\_BPF\_STACK}.
% However, this size is often insufficient for practical monitoring use cases.
% For instance, in the case of Memcached, existing monitoring tools collect metrics that require 656 bytes per instance, which already exceeds the default stack limit.
% Moreover, as the number of monitored microservices increases, the memory required to store their corresponding metrics grows proportionally, further stressing the BPF stack usage.

% eBPF does not provide general-purpose helper functions for dynamically allocating temporary local memory, as eBPF verifier must statically prove the memory safety of all accesses.
% Unlike the stack, which has a fixed size and a well-defined layout, dynamically allocated memory is difficult to verify due to uncertain lifetimes, pointer aliasing, and type ambiguity.
% Therefore, it is difficult to introduce custom helper functions that extend local memory, as their accesses cannot be precisely verified at verification time.

% To address this limitation while preserving memory safety guarantees, we increased the value of \texttt{MAX\_BPF\_STACK} in the Linux kernel, thereby enabling our monitoring system to allocate a larger BPF stack size sufficient to store the necessary metrics.


\subsection{Applying Memcached}
\label{subsec:case-study}
In this section, we present a case study to illustrate how {\sysname} collects both user and kernel metrics.  
We use Memcached metrics as a representative example of user metrics.  
% For kernel metrics, we focus on CPU-related statistics.
For kernel metrics, although {\sysname} supports various types, we focus on CPU-related statistics.

\textbf{Monitoring Memcached Metrics. }
Memcached is a widely used in-memory database in cloud environments.  
Due to its simplicity and high performance, as well as its support for distributed caching, it has been adopted in large-scale web services and microservice-based architectures.  
In traditional monitoring approaches, a monitoring process sends the \texttt{stats} command to Memcached.  
Upon receiving the request, Memcached invokes its internal metric collection functions---\texttt{server\_stats()} and \texttt{get\_stats()}---to selectively retrieve metrics from a set of predefined data structures.

These structures include seven types: \texttt{stats}, \texttt{stats\_state}, \texttt{rusage}, \texttt{slab\_stats}, \texttt{settings}, \texttt{thread\_stats}, and \texttt{itemstats\_t}.  
Although the combined size of these structures is approximately 7,440 bytes, the actual size of the metrics returned to the client is only about 656 bytes.  
Therefore, Memcached selectively collects only the required metrics from each structure.

In contrast, the proposed approach minimizes the involvement of Memcached during monitoring.  
Memcached only registers metrics information prior to monitoring; it plays no active role during runtime metric collection.  
% Moreover, {\sysname} rearranges the member fields of the metric-holding structures so that monitored metrics are placed at the beginning and unmonitored ones at the end.  
% This layout allows BPF programs to collect metrics efficiently, thereby collecting the necessary metrics with minimal overhead.
Moreover, {\sysname} rearranges the fields of the metrics structure so that only the required metrics are placed consecutively at the beginning of the structure.
This allows efficient retrieval of the necessary metrics in a single access.

\textbf{Monitoring CPU Metrics. }
For collecting CPU metrics, both conventional methods and {\sysname} retrieve metrics via \texttt{procfs}.
In traditional approaches, a system call is issued to access \texttt{procfs}, resulting in user–kernel context switch overhead.
In contrast, {\sysname} invokes \texttt{procfs} from within a BPF helper function, thereby eliminating the overhead of context switches.

Specifically, BPF helper calls \texttt{part\_stat\_read\_all()}, a function defined inside \texttt{procfs}.
This function internally invokes a series of other functions, eventually calling the macro \texttt{RELOC\_HIDE} to access CPU metrics.
Throughout this entire code path, no locks are acquired.
As a result, {\sysname} ensures safe monitoring from within XDP, without introducing any blocking due to lock contention.
% \section{Case Study}
\label{sec:case-study}
In this section, we present a case study to illustrate how {\sysname} collects both user and kernel metrics.  
We use Memcached metrics as a representative example of user metrics.  
For kernel metrics, we focus on CPU-related statistics.

\textbf{Monitoring Memcached Metrics }
Memcached is a widely used in-memory database in cloud environments.  
Due to its simplicity and high performance, as well as its support for distributed caching, it has been adopted in large-scale web services and microservice-based architectures.  
In traditional monitoring approaches, a monitoring process sends the \texttt{stats} command to Memcached.  
Upon receiving the request, Memcached invokes its internal metric collection functions—\texttt{server\_stats()} and \texttt{get\_stats()}—to selectively retrieve metrics from a set of predefined data structures.

These structures include seven types: \texttt{stats}, \texttt{stats\_state}, \texttt{rusage}, \texttt{slab\_stats}, \texttt{settings}, \texttt{thread\_stats}, and \texttt{itemstats\_t}.  
Although the combined size of these structures is approximately 7,440 bytes, the actual size of the metrics returned to the client is only about 656 bytes.  
Therefore, Memcached selectively collects only the required metrics from each structure.

This design introduces two drawbacks in traditional monitoring: (1) metric collection interferes with the execution of Memcached itself, and (2) additional overhead occurs due to frequent user–kernel context switches.

In contrast, the proposed approach minimizes the involvement of Memcached during monitoring.  
Memcached only registers metadata for the metrics prior to monitoring; it plays no active role during runtime metric collection.  
Moreover, we rearrange the member variables of the metric-holding structures so that monitored metrics are placed at the beginning and unmonitored ones at the end.  
This layout allows BPF programs to efficiently copy only the monitored portion—starting from the beginning of the structure—into the BPF stack, thereby collecting the necessary metrics with minimal overhead.

\textbf{Monitoring CPU Metrics }
For collecting CPU metrics, both conventional methods and {\sysname} retrieve metrics via \texttt{procfs}.
In traditional approaches, a system call is issued to access \texttt{procfs}, resulting in user–kernel context switch overhead.
In contrast, {\sysname} invokes \texttt{procfs} from within a BPF helper function, thereby eliminating the overhead of context switches.

Specifically, the BPF helper calls \texttt{bpf\_get\_user\_cpu\_metrics()}, a function defined inside \texttt{procfs}.
This function internally invokes a series of other functions, eventually calling the macro \texttt{RELOC\_HIDE} to access CPU metrics.
Throughout this entire code path, no locks are acquired.
As a result, {\sysname} ensures safe monitoring from within XDP, without introducing any blocking due to lock contention.
\section{Evaluation}
% \label{sec:evaluation}

% \begin{figure*}[t]  % Use figure* for spanning across both columns
%     \centering
%     % First image (a)
%     \begin{minipage}[b]{0.33\textwidth}  % Adjust width as needed
%         \centering
%         \includegraphics[width=\textwidth]{SoCC-2025/figures/experiment/cdf/instance1-kernel.png}
%         \subcaption{1 instance}
%         \label{fig:instance1-kernel}
%     \end{minipage}
%     % \hfill
%     % Second image (b)
%     \begin{minipage}[b]{0.33\textwidth}  % Adjust width as needed
%         \centering
%         \includegraphics[width=\textwidth]{SoCC-2025/figures/experiment/cdf/instance5-kernel.png}
%         \subcaption{5 instances}
%         \label{fig:instance5-kernel}
%     \end{minipage}
%     % \hfill
%     % Third image (c)
%     \begin{minipage}[b]{0.33\textwidth}  % Adjust width as needed
%         \centering
%         \includegraphics[width=\textwidth]{SoCC-2025/figures/experiment/cdf/instance10-kernel.png}
%         \subcaption{10 instances}
%         \label{fig:instance10-kernel}
%     \end{minipage}
    
%     % \hfill
%     \begin{minipage}[b]{0.33\textwidth}  % Adjust width as needed
%         \centering
%         \includegraphics[width=\textwidth]{SoCC-2025/figures/experiment/cdf/interval1000-kernel.png}
%         \subcaption{interval 1000 ms}
%         \label{fig:interval1000-kernel}
%     \end{minipage}
%     % \hfill
%     % Second image (b)
%     \begin{minipage}[b]{0.33\textwidth}  % Adjust width as needed
%         \centering
%         \includegraphics[width=\textwidth]{SoCC-2025/figures/experiment/cdf/interval100-kernel.png}
%         \subcaption{interval 100 ms}
%         \label{fig:interval100-kernel}
%     \end{minipage}
%     % \hfill
%     % Third image (c)
%     \begin{minipage}[b]{0.33\textwidth}  % Adjust width as needed
%         \centering
%         \includegraphics[width=\textwidth]{SoCC-2025/figures/experiment/cdf/interval10-kernel.png}
%         \subcaption{interval 10 ms} 
%         \label{fig:interval10-kernel}
%     \end{minipage}
%     % Overall caption
%     \caption{Monitoring latency of kernel metrics under varying numbers of Memcached instances and sampling intervals}
%     \label{fig:monitor-latency-kernel}
% \end{figure*}



% \begin{figure*}[t]  % Use figure* for spanning across both columns
%     \centering
%     % First image (a)
%     \begin{minipage}[b]{0.33\textwidth}  % Adjust width as needed
%         \centering
%         \includegraphics[width=\textwidth]{SoCC-2025/figures/experiment/cdf/instance1-user.png}
%         \subcaption{1 instance}
%         \label{fig:instance1-user}
%     \end{minipage}
%     % Second image (b)
%     \begin{minipage}[b]{0.33\textwidth}  % Adjust width as needed
%         \centering
%         \includegraphics[width=\textwidth]{SoCC-2025/figures/experiment/cdf/instance5-user.png}
%         \subcaption{5 instances}
%         \label{fig:instance5-user}
%     \end{minipage}
%     % Third image (c)
%     \begin{minipage}[b]{0.33\textwidth}  % Adjust width as needed
%         \centering
%         \includegraphics[width=\textwidth]{SoCC-2025/figures/experiment/cdf/instance10-user.png}
%         \subcaption{10 instances}
%         \label{fig:instance10-user}
%     \end{minipage}
    
%     \begin{minipage}[b]{0.33\textwidth}  % Adjust width as needed
%         \centering
%         \includegraphics[width=\textwidth]{SoCC-2025/figures/experiment/cdf/interval1000-user.png}
%         \subcaption{interval 1000 ms}
%         \label{fig:interval1000-user}
%     \end{minipage}
%     % Second image (b)
%     \begin{minipage}[b]{0.33\textwidth}  % Adjust width as needed
%         \centering
%         \includegraphics[width=\textwidth]{SoCC-2025/figures/experiment/cdf/interval100-user.png}
%         \subcaption{interval 100 ms}
%         \label{fig:interval100-user}
%     \end{minipage}
%     % Third image (c)
%     \begin{minipage}[b]{0.33\textwidth}  % Adjust width as needed
%         \centering
%         \includegraphics[width=\textwidth]{SoCC-2025/figures/experiment/cdf/interval10-user.png}
%         \subcaption{interval 10 ms} 
%         \label{fig:interval10-user}
%     \end{minipage}
%     % Overall caption
%     \caption{Monitoring latency of user metrics under varying numbers of Memcached instances and sampling intervals}
%     \label{fig:monitor-latency-user}
% \end{figure*}





% \begin{figure*}[t]
%   \centering
%   % --- 左 ---
%   \begin{minipage}{0.48\linewidth}
%     \centering
%     \includegraphics[width=\linewidth]{graph/APP_Throughput/throughput_user.pdf}
%     \caption{Memcached throughput while monitoring kernel metrics.}
%     \label{fig:throughput_user}
%   \end{minipage}
%   \hfill
%   % --- 右 ---
%   \begin{minipage}{0.48\linewidth}
%     \centering
%     \includegraphics[width=\linewidth]{graph/APP_Throughput/throughput_kernel.pdf}
%     \caption{Memcached throughput while monitoring user metrics.}
%     \label{fig:throughput_kernel}
%   \end{minipage}
% \end{figure*}

\begin{figure*}[t]
    \centering
    \includegraphics[width=0.93\linewidth]{graph/APP_Throughput/throughput_user.pdf}
    \caption{Memcached throughput while monitoring user metrics.}
    \label{fig:throughput_user}
\end{figure*}

\begin{figure*}[t]
    \centering
    \includegraphics[width=0.93\linewidth]{graph/APP_Throughput/throughput_kernel.pdf}
    \caption{Memcached throughput while monitoring kernel metrics.}
    \label{fig:throughput_user}
\end{figure*}

\begin{figure*}[t]
    \centering
    \includegraphics[width=0.93\linewidth]{graph/APP_Latency/latency99_user.pdf}
    \caption{Memcached 99th latency while monitoring user metrics.}
    \label{fig:throughput_user}
\end{figure*}

\begin{figure*}[t]
    \centering
    \includegraphics[width=0.93\linewidth]{graph/APP_Latency/latency99_kernel.pdf}
    \caption{Memcached 99th latency while monitoring kernel metrics.}
    \label{fig:throughput_user}
\end{figure*}

To evaluate {\sysname}, we conducted some experiments.
Sec.~\ref{subsec:evaluation-setup} describes the experimental setup.
We analyze how much it reduces monitoring latency in Sec.~\ref{subsec:monitoring-latency}.
We then analyze how effectively {\sysname} mitigates monitoring overhead in Sec.~\ref{subsec:monitoring-overhead}.
Finally, in Sec.~\ref{subsec:priority}, we investigate the monitoring latency and overhead when the traditional monitoring system is prioritized and compare the results with those of {\sysname}.

\subsection{Evaluation Setup}
\label{subsec:evaluation-setup}

% \begin{table}[ht]
% \centering
% % \small 
% \scriptsize
% \caption{Monitoring Server Configuration}
% \begin{tabular}{|l|l|}
% \hline
% OS     & Ubuntu 24.04 (Linux kernel v6.8)                  \\ \hline
% CPU    & Intel(R) Xeon(R) Gold 5418Y (48 cores)\\ \hline
% NIC    & Intel Corporation Ethernet Controller X710 for 10GbE SFP+ \\ \hline
% DRAM    & 256 GB  \\ \hline     
% \end{tabular}
% \label{tab:system-config-server}
% \end{table}

% \begin{table}[ht]
% \centering
% % \small 
% \scriptsize
% \caption{Monitoring Client Configuration}
% \begin{tabular}{|l|l|}
% \hline
% OS     & Ubuntu 22.04 (Linux kernel v5.15)                  \\ \hline
% CPU    & Intel(R) Xeon(R) E-2276G @ 3.80GHz (6 cores) \\ \hline
% NIC    & Intel Corporation Ethernet Controller X710 for 10GbE SFP+ \\ \hline
% DRAM    & 32 GB  \\ \hline     
% \end{tabular}
% \label{tab:system-config-client}
% \end{table}

\begin{table}[t]
\centering
% \small
\scriptsize
\caption{Experimental Setup}
\begin{tabular}{|l|p{3cm}|p{3cm}|}
% \begin{tabular}{|l|l|l|}
\hline
       & \textbf{Monitoring Server} & \textbf{Monitoring Client} \\ \hline
OS     & Ubuntu 24.04 (Linux kernel v6.8) & Ubuntu 22.04 (Linux kernel v5.15) \\ \hline
CPU    & Intel(R) Xeon(R) Gold 5418Y @ 2.00GHz (48 cores, 2 sockets) & Intel(R) Xeon(R) E-2276G @ 3.80GHz (6 cores, 1 socket) \\ \hline
NIC    & Intel Ethernet Controller X710 for 10GbE SFP+ & Intel Ethernet Controller X710 for 10GbE SFP+ \\ \hline
DRAM   & 256 GB DDR5-4400 & 32 GB DDR4-2666 \\ \hline
\end{tabular}
\label{tab:system-config}
\end{table}

We use Netdata\cite{netdata}, as a comparison system.
Netdata is widely used by cloud providers such as AWS and Microsoft Azure~\cite{netdata}.
The monitoring process of Netdata is executed in user space.
% We use YCSB~\cite{ycsb} as a benchmark to perform update and insert operations on a Memcached server~\cite{memcached}.
% In this experiment, we configure YCSB with a Zipfian request distribution, which generates requests that follow a skewed access pattern commonly seen in real-world workloads.
% The database is initialized with a single record (recordcount=1), consisting of one field (fieldcount=1) of 100,000 bytes in length (fieldlength=100000).
We use YCSB~\cite{ycsb} as a benchmark to perform update and insert operations on a Memcached server~\cite{memcached}.
In this experiment, YCSB is configured to generate requests following a Zipfian distribution.
The monitoring server and the monitoring client are directly connected without a switch.
% The Memcached server, the monitoring system, and the YCSB benchmark all run on the same physical server.
% To minimize interference, we isolate their CPU cores and NUMA nodes.
% Specifically, both Memcached and Netdata are pinned to core 0 of NUMA node 0.
% YCSB is configured to launch as many threads as the number of cores in NUMA node 1.
The experimental environment is shown in Tab.~\ref{tab:system-config}.

As discussed in Sec.~\ref{subsec: cloud-monitoring}, cloud service monitoring requires low latency and minimal interference with the host services.
To confirm {\sysname} meets these requirements, we evaluate it from the following aspects.
\begin{itemize}
    \item \textit{Monitoring Latency}: 
    Monitoring latency is the duration between when the monitoring client sends a packet to the monitored server and when it receives a reply from that server.
    % \item \textit{CPU Utilization}: 
    % CPU utilization represents the percentage of CPU resources consumed by the monitoring system. 
    % We measure these metrics using \texttt{pidstat}~\cite{pidstat} for Netdata, which tracks process-level CPU usage. 
    % For {\sysname}, CPU utilization is measured based on \texttt{RDTSC}~\cite{rdtsc}, enabling fine-grained cycle-level accounting.
    % These metrics are used to evaluate the computational overhead introduced by each monitoring approach.
    \item \textit{CPU Utilization}: 
    To evaluate the computational overhead introduced by each monitoring approach, we measure CPU utilization, which represents the percentage of CPU resources consumed by the monitoring system. 
    We use \texttt{pidstat}~\cite{pidstat} to measure process-level CPU usage for Netdata. 
    For {\sysname}, CPU utilization is measured based on \texttt{RDTSC}~\cite{rdtsc}, enabling fine-grained cycle-level accounting.
    % \item \textit{Memcached Throughput}: 
    % Memcached throughput is the number of operations per second.
    % The Memcached throughput reflects the performance of the host's services. 
    % Therefore, it is used to evaluate how much monitoring interferes with the host's performance.
    \item \textit{Memcached Throughput}:  
    To evaluate how much monitoring interferes with the host's performance, we measure Memcached throughput, defined as the number of operations per second.  
    Memcached throughput reflects the performance of the host's services and is used as an indicator of the impact caused by monitoring.
\end{itemize}

We evaluate the impact of the following two parameters on monitoring performance to investigate whether real-time metrics can be collected under high-load conditions.
\begin{itemize}
    \item \textit{Number of Memcached Instances}: As mentioned in Sec.~\ref{sec:background}, many applications are deployed in distributed systems. 
    We evaluate how monitoring performance changes as the number of Memcached instances increases and the load intensifies.
    The number of instances varies from 1 to 10.
    \item \textit{Sampling Interval}: The required sampling interval is determined by the updating frequency of metrics.
    The default interval is 1000 ms in Netdata and it is configurable.
    We evaluate monitoring performance with 10 - 1000\textit{ms} sampling intervals.
\end{itemize}


\subsection{Monitoring Latency}
\label{subsec:monitoring-latency}

% \begin{table}[t]
%     \centering
%     % \caption{Latency statistics for different monitoring intervals (unit: ms)}
%     \caption{Latency statistics for monitoring 10 Memcached instances at different intervals (unit: ms)}
%     \label{tab:latency-stats}
%     \small
%     \begin{tabular}{lccc}
%         \toprule
%          & 1000 ms & 100 ms & 10 ms \\
%         \midrule
%         Mean latency        & 1.040 & 1.192 & 1.058 \\
%         50th percentile     & 1.050 & 0.974 & 0.839 \\
%         90th percentile     & 1.505 & 1.553 & 1.444 \\
%         99th percentile     & 1.611 & 5.182 & 5.776 \\
%         Minimum latency     & 0.592 & 0.551 & 0.399 \\
%         Maximum latency     & 1.641 & 10.790 & 9.554 \\
%         Standard deviation  & 0.320 & 0.943 & 0.942 \\
%         \bottomrule
%     \end{tabular}
% \end{table}

% \begin{table}[htbp]
%     \centering
%     \caption{Latency statistics for monitoring Memcached with different instance counts (interval = 10 ms, unit: ms)}
%     \label{tab:latency-10ms}
%     \small
%     \begin{tabular}{lccc}
%         \toprule
%                       & 1 instance & 5 instances & 10 instances \\
%         \midrule
%         Mean latency        & 0.993 & 0.988 & 1.058 \\
%         50th percentile     & 0.786 & 0.809 & 0.839 \\
%         90th percentile     & 1.392 & 1.385 & 1.444 \\
%         99th percentile     & 5.405 & 4.946 & 5.776 \\
%         Minimum latency     & 0.398 & 0.396 & 0.399 \\
%         Maximum latency     & 10.625 & 10.656 & 9.554 \\
%         Standard deviation  & 0.890 & 0.811 & 0.942 \\
%         \bottomrule
%     \end{tabular}
% \end{table}


% \begin{table*}[htbp]
%     \centering
%     \caption{Latency statistics for monitoring user metrics under different sampling intervals (unit: ms)}
%     \label{tab:latency-user-metrics}
%     \small
%     \begin{tabular}{lccc|ccc}
%         \toprule
%          & \multicolumn{3}{c}{{\sysname}} & \multicolumn{3}{c}{Netdata} \\
%         \cmidrule(lr){2-4} \cmidrule(lr){5-7}
%         & 1000 ms & 100 ms & 10 ms & 1000 ms & 100 ms & 10 ms \\
%         \midrule
%         Mean latency        & $5.96\times 10^{-2}$ & $5.5\times 10^{-2}$ & $5.1\times 10^{-2}$ & $1.0$   & $1.2$   & $1.1$   \\
%         % 50th percentile     & $5.9\times 10^{-2}$  & $5.4\times 10^{-2}$ & $5.0\times 10^{-2}$ & $1.1$   & $9.7\times 10^{-1}$ & $8.4\times 10^{-1}$ \\
%         % 90th percentile     & $6.2\times 10^{-2}$  & $5.9\times 10^{-2}$ & $5.2\times 10^{-2}$ & $1.5$   & $1.6$   & $1.4$   \\
%         99th percentile     & $6.4\times 10^{-2}$  & $6.6\times 10^{-2}$ & $5.4\times 10^{-2}$ & $1.6$   & $5.2$   & $5.8$   \\
%         Minimum latency     & $5.0\times 10^{-2}$  & $4.4\times 10^{-2}$ & $4.1\times 10^{-2}$ & $5.9\times 10^{-1}$ & $5.5\times 10^{-1}$ & $4.0\times 10^{-1}$ \\
%         Maximum latency     & $6.5\times 10^{-2}$  & $6.9\times 10^{-2}$ & $5.9\times 10^{-2}$ & $1.6$   & $1.1\times 10^{1}$ & $9.6$   \\
%         Standard deviation  & $2.2\times 10^{-3}$  & $3.4\times 10^{-3}$ & $1.5\times 10^{-3}$ & $3.2\times 10^{-1}$ & $9.4\times 10^{-1}$ & $9.4\times 10^{-1}$ \\
%         \bottomrule
%     \end{tabular}
% \end{table*}

% \begin{table*}[htbp]
%     \centering
%     \caption{Latency statistics for monitoring user metrics under different sampling interlals (unit: $\mu$s)}
%     \label{tab:latency-table}
%     \small
%     \begin{tabular}{lccc|ccc}
%         \toprule
%          & \multicolumn{3}{c}{{\sysname}} & \multicolumn{3}{c}{Netdata} \\
%         \cmidrule(lr){2-4} \cmidrule(lr){5-7}
%         & 1000 ms & 100 ms & 10 ms & 1000 ms & 100 ms & 10 ms \\
%         \midrule
%         Mean latency        & 59.6 & 54.8 & 50.5 & 1040 & 1190 & 1060 \\
%         % 50th percentile     & 59.4 & 54.0 & 50.4 & 1050 & 974  & 839 \\
%         % 90th percentile     & 62.0 & 59.2 & 52.3 & 1505 & 1553 & 1444 \\
%         99th percentile     & 64.4 & 66.4 & 54.3 & 1611 & 5182 & 5776 \\
%         Minimum latency     & 50.3 & 43.8 & 40.8 & 592  & 551  & 399 \\
%         Maximum latency     & 64.6 & 69.0 & 58.6 & 1641 & 10790 & 9554 \\
%         Standard deviation  & 2.17 & 3.41 & 1.45 & 320  & 943  & 942 \\
%         \bottomrule
%     \end{tabular}
% \end{table*}

We evaluate the latency of {\sysname} in monitoring metrics of kernel and user while performing update and insert operations on the Memcached server using YCSB. 
We monitored CPU utilization as kernel metrics and Memcached metrics as user metrics.


% Tab.~\ref{tab:monitoring-latency} summarizes the monitoring latency of Netdata and {\sysname}.
% {\sysname} reduces the maximum latency by two orders of magnitude compared with Netdata.
% High load causes a delay in monitoring messages, with the maximum latency increasing rapidly in Netdata.
% Not only does {\sysname} significantly reduce the maximum latency, but it also achieves a minimum latency that is one order of magnitude lower than that of Netdata.
% This difference arises because Netdata collects and communicates metrics using the HTTP protocol, whereas {\sysname} operates at the NIC driver layer.
% As a result, the baseline latency of {\sysname} is inherently lower.

% Fig.~\ref{fig:monitor-latency-kernel} and Fig.~\ref{fig:monitor-latency-user} show the CDFs of monitoring latency for kernel and user metrics, respectively, across different numbers of Memcached instances and sampling intervals.

\textbf{Monitoring Latency for Kernel Metrics.}
Fig.~\ref{fig:instance1-kernel}, Fig.~\ref{fig:instance5-kernel}, and Fig.~\ref{fig:instance10-kernel} illustrate the monitoring latency of {\sysname} and Netdata with varying sampling intervals under fixed numbers of Memcached instances.
Fig.~\ref{fig:instance1-kernel} shows the case where the number of instances is fixed at one. Across all sampling intervals (1000~ms, 100~ms, and 10~ms), {\sysname} consistently achieves lower latency, with the 99th percentile latency being 81.64~$\mu\text{s}$, 74.0~$\mu\text{s}$, and 70.3~$\mu\text{s}$, respectively. In contrast, Netdata exhibits significantly higher latency, with 99th percentile values of 3.17~ms, 4.83~ms, and 5.41~ms for the same intervals.
Fig.~\ref{fig:instance5-kernel} presents the results for five instances. Again, {\sysname} maintains low latency across all intervals. While Netdata does not show a sharp increase in tail latency at 1000~ms (99th percentile: 1.63~ms), the tail latency becomes substantially larger at 100~ms and 10~ms (99th percentile: 5.01~ms and 4.96~ms, respectively).
Similar trends are observed in Fig.~\ref{fig:instance10-kernel}, where the number of instances is fixed at ten. {\sysname} keeps the latency low for all sampling intervals (99th percentile: 75.2~$\mu\text{s}$, 71.3~$\mu\text{s}$, and 70.1~$\mu\text{s}$), while Netdata again suffers from increased tail latency, particularly at 100~ms and 10~ms intervals (99th percentile: 5.18~ms and 5.77~ms, respectively).
These results indicate that {\sysname} enables consistently low-latency monitoring for Memcached servers with varying numbers of instances, reducing tail latency by approximately two orders of magnitude compared to Netdata.

Fig.~\ref{fig:interval1000-kernel}, Fig.~\ref{fig:interval100-kernel}, and Fig.~\ref{fig:interval10-kernel} show the monitoring latency of kernel metrics when the sampling interval is fixed and the number of Memcached instances is varied.
In Fig.~\ref{fig:interval1000-kernel}, with the sampling interval fixed at 1000~ms, {\sysname} shows consistently low latency across all instance counts ($\le 81.1~\mu\text{s}$). Netdata, on the other hand, exhibits a sharp increase in tail latency when the number of instances is one. Although the increase is less pronounced for five and ten instances, Netdata's latency remains higher than that of {\sysname}.
Fig.~\ref{fig:interval100-kernel} shows results with a 100~ms interval. {\sysname} again maintains low latency across all instance counts ($\le 74.0~\mu\text{s}$), while Netdata shows significantly higher latency and larger tail values.
In Fig.~\ref{fig:interval10-kernel}, the sampling interval is fixed at 10~ms. {\sysname} continues to provide low monitoring latency ($\le 71.2~\mu\text{s}$) regardless of the number of instances, whereas Netdata suffers from high tail latency.
These results confirm that {\sysname} achieves low-latency monitoring across various sampling intervals, significantly reducing tail latency by up to two orders of magnitude compared to Netdata.
In all cases, the monitoring latency was shorter than the sampling interval.

\textbf{Monitoring Latency for User Metrics.} 
To retrieve user metrics, {\sysname} is required to access metrics in user space from the kernel via XDP\@. 
% To achieve low-latency monitoring in this setting, {\sysname} is designed to access user metrics through shared memory, enabling monitoring performance comparable to that of kernel metric collection.
{\sysname} uses shared memory to access user metrics to avoid context switches involved in invoking user-level APIs.  
% We now evaluate whether this approach indeed achieves low-latency monitoring comparable to that for kernel metrics.
Here we demonstrate that approach achieves low-latency in monitoring.

Fig.~\ref{fig:instance1-user}, Fig.~\ref{fig:instance5-user}, and Fig.~\ref{fig:instance10-user} show the monitoring latency with a fixed number of Memcached instances under varying sampling intervals.
Fig.~\ref{fig:instance1-user} illustrates the case with one instance, where the sampling interval is varied.
Across all sampling intervals, {\sysname} consistently achieves low-latency monitoring.
The monitoring latencies for sampling intervals of 1000, 100, and 10~ms were 152~$\mu$s, 145~$\mu$s, and 130~$\mu$s, respectively.
Although this is approximately twice the latency observed in kernel monitoring under the same conditions, the latency remains low (81.6~$\mu$s, 74.0~$\mu$s, and 70.3~$\mu$s, respectively, for kernel metrics).
In contrast, Netdata exhibits significantly higher tail latency, with monitoring latencies of 4.23~ms, 5.54~ms, and 5.07~ms for the same sampling intervals.
Even when the number of instances is fixed at five (Fig.~\ref{fig:instance5-user}), {\sysname} maintains low monitoring latency ($\le$197~$\mu$s).
On the other hand, Netdata shows much higher latency.
This reflects a trend where shorter sampling intervals exacerbate Netdata's tail latency.
Specifically, the 99.9th percentile latencies for sampling intervals of 1000, 100, and 10~ms were 3.11~ms, 5.29~ms, and 5.73~ms, respectively.
Similar behavior is observed when the number of instances is increased to ten (Fig.~\ref{fig:instance10-user}).
{\sysname} consistently maintains low monitoring latency across all sampling intervals, while Netdata continues to exhibit high latency.
These results demonstrate that {\sysname} can achieve low-latency monitoring of user metrics regardless of the number of instances.

Fig.~\ref{fig:interval1000-user}, Fig.~\ref{fig:interval100-user}, and Fig.~\ref{fig:interval10-user} show monitoring latency with fixed sampling intervals while varying the number of instances.
As shown in Fig.~\ref{fig:interval1000-user}, {\sysname} achieves low latency across all instance counts ($\le$197~$\mu$s), whereas Netdata shows significantly higher latency ($\le$5.23~ms).
The same trend is observed when the interval is fixed at 100~ms (Fig.~\ref{fig:interval100-user}), where {\sysname} maintains latency $\le$187~$\mu$s, while Netdata reaches up to $\le$7.8~ms.
Even in the 10~ms interval case (Fig.~\ref{fig:interval10-user}), {\sysname} maintains low latency ($\le$173~$\mu$s) regardless of the instance count, whereas Netdata exhibits increased latency.
These results confirm that {\sysname} provides low-latency monitoring for user metrics under all sampling intervals and instance counts.
In all cases, similar to kernel metric monitoring, the monitoring latency remains shorter than the sampling interval, indicating that the monitoring process can be completed within each interval.


% Tab.~\ref{tab:latency-table} presents various latency statistics for the case where the number of Memcached instances is fixed at 10 and the sampling interval is varied.
% The mean latency remains nearly constant regardless of the sampling interval for both {\sysname} and Netdata, while {\sysname} consistently achieves a latency that is approximately 1/20 that of Netdata.
% The 99th percentile latency of {\sysname} does not show growth as the interval decreases from 1000ms to 10ms (from 64.4~$\mu$s to 54.3~$\mu$s, a 15\% decrease). In contrast, Netdata’s 99th percentile latency increases by 258.2\% (from 1611~$\mu$s to 5776~$\mu$s).
% The minimum latency remains stable across sampling intervals for both systems. Nevertheless, {\sysname}’s minimum latency is an order of magnitude smaller than that of Netdata.
% For the maximum latency, {\sysname} shows no increase with shorter intervals, whereas Netdata experiences a 482.3\% increase (from 1641~$\mu$s to 9554~$\mu$s).
% Regarding standard deviation, Netdata’s latency variance is two orders of magnitude larger than that of {\sysname}, regardless of the sampling interval.
% These results indicate that {\sysname} is less affected by sampling interval changes and maintains consistently low and stable latency across all sampling intervals, unlike Netdata.


% Fig.~\ref{fig:instance1-kernel}, Fig.~\ref{fig:instance5-kernel}, and Fig.~\ref{fig:instance10-kernel} illustrate the monitoring latency of {\sysname} and Netdata with varying sampling intervals under fixed numbers of Memcached instances.
% Fig.~\ref{fig:instance1-kernel} shows the case where the number of instances is fixed at one. Across all sampling intervals (1000~ms, 100~ms, and 10~ms), {\sysname} consistently achieves lower latency, with the 99th percentile latency being 81.64~$\mu$s, 74.0~$\mu$s, and 70.3~$\mu$s, respectively. In contrast, Netdata exhibits significantly higher latency, with 99th percentile values of 3.17~ms, 4.83~ms, and 5.41~ms for the same intervals.
% Fig.~\ref{fig:instance5-kernel} presents the results for five instances. Again, {\sysname} maintains low latency across all intervals. While Netdata does not show a sharp increase in tail latency at 1000~ms (99th percentile: 1.63~ms), the tail latency becomes substantially larger at 100~ms and 10~ms (99th percentile: 5.01~ms and 4.96~ms, respectively).
% Similar trends are observed in Fig.~\ref{fig:instance10-kernel}, where the number of instances is fixed at ten. {\sysname} keeps the latency low for all sampling intervals (75.2~$\mu$s, 71.3~$\mu$s, and 70.1~$\mu$s), while Netdata again suffers from increased tail latency, particularly at 100~ms and 10~ms intervals (99th percentile: 5.18~ms and 5.77~ms, respectively).
% These results indicate that {\sysname} enables consistently low-latency monitoring for Memcached servers with varying numbers of instances, reducing tail latency by approximately two orders of magnitude compared to Netdata.

% Fig.~\ref{fig:interval1000-kernel}, Fig.~\ref{fig:interval100-kernel}, and Fig.~\ref{fig:interval10-kernel} show the monitoring latency of kernel metrics when the sampling interval is fixed and the number of Memcached instances is varied.
% In Fig.~\ref{fig:interval1000-kernel}, with the sampling interval fixed at 1000~ms, {\sysname} shows consistently low latency across all instance counts (<= 81.1~$\mu$s). Netdata, on the other hand, exhibits a sharp increase in tail latency when the number of instances is one. Although the increase is less pronounced for five and ten instances, Netdata's latency remains higher than that of {\sysname}.
% Fig.~\ref{fig:interval100-kernel} shows results with a 100~ms interval. {\sysname} again maintains low latency across all instance counts (<= 74.0~$\mu$s), while Netdata shows significantly higher latency and larger tail values.
% In Fig.~\ref{fig:interval10-kernel}, the sampling interval is fixed at 10~ms. {\sysname} continues to provide low monitoring latency (<= 71.2~$\mu$s) regardless of the number of instances, whereas Netdata suffers from high tail latency.
% These results confirm that {\sysname} achieves low-latency monitoring across various sampling intervals, significantly reducing tail latency by up to two orders of magnitude compared to Netdata.

% \textbf{Latency vs Instances. }
% As Fig.~\ref{fig:kernel_latency_a} and ~\ref{fig:user_latency_a} show, Netdata exhibits high monitoring latency regardless of the number of Memcached instances.
% The experiment is conducted with a sampling interval fixed to 10 ms.
% For kernel monitoring, the monitoring latency remains consistently high regardless of the number of Memcached instances, with values of 7.64 ms, 7.12 ms, and 7.62 ms observed for 1, 5, and 10 instances, respectively.
% Similarly, for user metrics monitoring, the latency remains high at 7.04 ms, 7.43 ms, and 6.74 ms.
% Under high-load conditions, CPU contention becomes a critical factor, causing delays in scheduling Netdata processes.

% In contrast, {\sysname} demonstrates remarkable stability in latency, even as the number of instances increases.
% The latency of kernel monitoring remains low and stable, measuring 72.42 µs with 1 instance, 75.93 µs with 5 instances, and 74.34 µs with 10 instances.
% Similarly, for user monitoring, the latency stays low at 49.53 µs, 53.20 µs, and 56.04 µs.
% This stability can be attributed to {\sysname}'s architecture, which ensures that it is not subject to traditional process scheduling.
% Instead, it is executed immediately after a hardware interrupt, minimizing delays.

% \textbf{Latency vs Interval. }
% As Fig.~\ref{fig:kernel_latency_b} and Fig~\ref{fig:user_latency_b} show, Netdata's monitoring latency increases as the sampling interval becomes shorter.
% The experiment is conducted with the number of Memcached instances fixed at 10.
% As the interval shortens from 1000 ms to 100 ms and then to 10 ms, Netdata's latency of kernel monitoring increases to 1.593 ms, 7.196 ms, and 7.624 ms, respectively.
% Netdata's latency of user monitoring also rises from 2.128 ms, 5.376 ms, and 6.748 ms.
% The relationship between the interval and the latency can be interpreted as follows.
% As the interval shortens, the frequency of heavy TCP/IP processing and user/kernel context switches required for monitoring also increases.
% This additional load from the monitoring itself leads to resource contention, making it more prone to delays.
% Furthermore, shorter intervals heighten the likelihood of blocking states during TCP/IP processing due to contention for locks.

% In contrast, {\sysname} does not exhibit an increase in latency as the interval shortens.
% When the interval decreases from 1000 ms to 100 ms and then to 10 ms, the latency remains consistent, transitioning from 77.52 µs to 74.08 µs and 74.34 µs in kernel monitoring and from 64.60 µs to 68.76 µs and 56.04 µs in user monitoring.
% {\sysname} bypasses the heavy processing associated with TCP/IP and user/kernel context switching and relies solely on lightweight operations, such as XDP program and \texttt{procfs} access, which impose minimal load on the system, leading to stable latency.
% {\sysname} also ensures that packets rarely overlap during processing.
% Even with 10 ms intervals, it achieves a latency of approximately 0.07 ms per monitoring operation, allowing the next packet to arrive only after the current one has been fully processed.
% Consequently, {\sysname} avoids simultaneous handling of multiple packets, maintaining consistently low latency in its operations.

% \textbf{CDF of monitoring latency. }
% Fig.~\ref{fig:kernel_cdf} and Fig.~\ref{fig:user_cdf} show the CDF of monitoring latency for kernel metrics and user metrics, measured with 10 Memcached instances and a sampling interval of 10 ms.
% In both cases, Netdata's tail latency remains stable up to approximately the $98^{\mathrm{th}}$ percentile but then rises sharply.
% The $99^{\mathrm{th}}$ percentile latency reaches 2.458 ms for kernel monitoring and 2.545 ms for user monitoring, respectively.
% On the other hand, {\sysname}'s tail latency remains low and stable.
% The $99^{\mathrm{th}}$ percentile latency of kernel monitoring and user monitoring is 70.16 µs and 54.33 µs respectively.
% {\sysname} reduces the $99^{\mathrm{th}}$ percentile latency by four to two orders of magnitude compared to Netdata.

\subsection{Monitoring Interference}
\label{subsec:monitoring-overhead}

We evaluate the monitoring overhead based on two factors: the CPU utilization required for monitoring and the reduction in Memcached throughput on the host during monitoring.

\begin{table}[ht]
\centering
\small 
\caption{CPU Utilization vs Memcached instances (Monitoring Memcached Metrics, Interval is 10 ms)}
\begin{tabular}{|l|l|l|l|}
\hline
     & 1 instance & 5 instances  &  10 instances \\ \hline
 Netdata (plugin)   & 4.8 \% (0.12 \%)  & 4.9 \% (0.18 \%)  & 5.2 \% (0.24 \%) \\ \hline
 % Netdata plugin   & 0.12 \%  & 0.18 \%  & 0.24 \%  \\ \hline
 {\sysname}  & $4.7 \times 10^{-3}$ \% &  $1.7 \times 10^{-2}$ \%  & $4.7 \times 10^{-2}$ \% \\ \hline
\end{tabular}
\label{tab:cpu-util-instance}
\end{table}

\begin{table}[ht]
\centering
\small 
\caption{CPU Utilization vs Memcached interval (Monitoring Memcached Metrics, 10 instances)}
\begin{tabular}{|l|l|l|l|}
\hline
     & 1000 ms & 100 ms  &  10 ms \\ \hline
 Netdata (plugin)   & 1.1 \% (0.28 \%) & 1.8 \% (0.28 \%) & 5.2 \% (0.24 \%) \\ \hline
 % Netdata plugin   & 0.28 \%  & 0.28 \%  & 0.24 \%  \\ \hline
 {\sysname}  & $9.9 \times 10^{-4}$ \% &  $5.6 \times 10^{-3}$ \%  & $4.7 \times 10^{-2}$ \% \\ \hline
\end{tabular}
\label{tab:cpu-util-interval}
\end{table}



\textbf{CPU Utilization.}
We measure the CPU utilization of the monitoring server while it handles requests for Memcached metrics sent from the monitoring client.
These requests are issued to either Netdata or {\sysname}, and the server-side CPU usage incurred by each monitoring approach is recorded.
% The Netdata plugin refers to \texttt{go.d.plugin}, a child process of Netdata responsible for collecting user metrics.
Netdata has a child process called \texttt{go.d.plugin}, which is responsible for collecting user metrics.
This plugin queries the application’s metrics API and returns the collected values to the main Netdata process.
% The Netdata total CPU utilization represents the combined CPU usage of both the \texttt{go.d.plugin} and the main Netdata process.

% % Overall, {\sysname} consistently incurs significantly lower CPU utilization compared to Netdata, typically by one to four orders of magnitude. 
% As shown in Tables~\ref{tab:cpu-util-instance} and~\ref{tab:cpu-util-interval}, {\sysname} consistently incurs lower CPU utilization compared to Netdata—typically by one to four orders of magnitude.
% This efficiency stems from the architectural differences between the two systems: Netdata incurs substantial overhead due to context switches and processing within the network stack, whereas {\sysname} eliminates these components and operates in a much more lightweight manner.

Tab.~\ref{tab:cpu-util-instance} and Fig.~\ref{fig:cpu-util-instance} show the relationship between CPU utilization and the number of Memcached instances.
The Netdata CPU usage increases from 4.8\% to 5.2\%, corresponding to a 0.4 percentage point increase.
In contrast, {\sysname}’s CPU usage increases from $4.7 \times 10^{-3}$\% to $4.7 \times 10^{-2}$\%, corresponding to a 0.0423 percentage point increase.
While both systems show increasing CPU usage as the number of instances grows, {\sysname} maintains a lower usage level with a much smaller increase.
The plugin shows a more noticeable rise in CPU usage due to the growing number of API requests it sends to each Memcached instance.
Its CPU utilization increases from 0.12\% to 0.24\%.

Tab.~\ref{tab:cpu-util-interval} and Fig.~\ref{fig:cpu-util-interval} illustrate the relationship between CPU utilization and the monitoring sampling interval.
As the sampling interval decreases, the CPU usage increases for both Netdata and {\sysname}.
However, even at a sampling interval of 10~ms, {\sysname} maintains a CPU utilization that is two orders of magnitude lower than Netdata.
The CPU utilization of the plugin remains nearly constant, ranging from 0.24\% to 0.22\%.
This is because \texttt{go.d.plugin} is configured to send API requests to the application at a minimum interval of 1000ms.
Therefore, regardless of the sampling interval configured for Netdata, the plugin continues to collect metrics at a fixed 1000ms interval, resulting in negligible changes in its CPU usage.

% Overall, this efficiency stems from the architectural differences between the two systems: Netdata incurs substantial overhead due to context switches and processing within the network stack, whereas {\sysname} eliminates these components and operates in a much more lightweight manner.


% \begin{figure}[t]  % Use figure* for spanning across both columns
%     \centering
%     % First image (a)
%     \begin{minipage}[b]{0.7\columnwidth}  % Adjust width as need≤≤≤ed
%         \centering
%         \includegraphics[width=\textwidth]{SoCC-2025/figures/experiment/Memcached-throughput-k-met.png}
%         \subcaption{Memcached throughput while monitoring kernel metrics}
%         \label{fig:throughput-k-met}
%     \end{minipage}
%     \hfill
%     % Second image (b)
%     \begin{minipage}[b]{0.7\columnwidth}  % Adjust width as needed
%         \centering
%         \includegraphics[width=\textwidth]{SoCC-2025/figures/experiment/Memcached-throughput-u-met.png}
%         \subcaption{Memcached throughput while monitoring user metrics}
%         \label{fig:throughput-u-met}
%     \end{minipage}
%     \hfill
%     % Overall caption
%     \caption{Memcached throughput}
%     \label{fig:memcached-throughput}
% \end{figure}

% \begin{figure*}[t]
%   \centering

%   % First image (a)
%   \begin{minipage}[t]{0.36\textwidth}
%     \centering
%     \includegraphics[width=\textwidth]{SoCC-2025/figures/experiment/cdf-user-pri.png}
%     \subcaption{CDF of User Monitoring Latency}
%     \label{fig:cdf-u-met-pri}
%   \end{minipage}
%   \hspace{0.01\textwidth}
%   % Second image (b)
%   \begin{minipage}[t]{0.30\textwidth}
%     \centering
%     \includegraphics[width=\textwidth]{SoCC-2025/figures/experiment/elapsed-latency-nonpri.png}
%     \subcaption{Monitoring Latency of Non-Prioritized Netdata}
%     \label{fig:elapsed-latency-nonpri}
%   \end{minipage}
%   \hspace{0.01\textwidth}
%   % Third image (c)
%   \begin{minipage}[t]{0.30\textwidth}
%     \centering
%     \includegraphics[width=\textwidth]{SoCC-2025/figures/experiment/elapsed-latency-pri.png}
%     \subcaption{Monitoring Latency of Prioritized Netdata}
%     \label{fig:elapsed-latency-pri}
%   \end{minipage}

%   \caption{Monitoring latency comparison}
%   \label{fig:monitoring-latency-three}
% \end{figure*}


% \begin{figure}[t]
%     \centering
%     \includegraphics[width=0.8\columnwidth]{SoCC-2025/figures/experiment/Memcached-throughput-u-met-pri.png}
%     \caption{Memcached Throughput while Monitoring User Metrics}
%     \label{fig:throughput-u-met-pri}
% \end{figure}

\textbf{Impact of Monitoring on Host Service Performance. }
To evaluate the extent to which monitoring interferes with the host’s services, we examine the throughput of Memcached during monitoring.
Fig.~\ref{fig:throughput-k-met} and Fig.~\ref{fig:throughput-u-met} show the Memcached throughput on the host machine when monitoring kernel and user metrics, respectively.
These figures illustrate how throughput changes as the monitoring sampling interval is varied.
In both kernel and user monitoring, Netdata reduces Memcached throughput compared to the no-monitoring baseline.
Moreover, the shorter the sampling interval, the more throughput degrades.
At a 10 ms sampling interval, kernel monitoring with Netdata causes the Memcached throughput to drop by approximately 10.6\% (from 12.48 K ops/sec to 11.16 K ops/sec), and user monitoring results in an 11.8\% (from 12.48 K ops/sec to 11.01 K ops/sec) decrease.
This performance degradation correlates with the increased CPU utilization of Netdata shown in Fig.~\ref{fig:cpu-util}, suggesting that CPU resource contention caused by Netdata interferes with the execution of host services.
In contrast, {\sysname} maintains nearly the same Memcached throughput regardless of the sampling interval.
As shown in Fig.~\ref{fig:cpu-util} and Fig.~\ref{fig:memcached-throughput}, {\sysname} is lightweight and introduces minimal interference, thereby preserving the performance of the Memcached service.


\subsection{Comparison with Prioritized Netdata}
\label{subsec:priority}

We investigate the relationship between monitoring latency and Memcached throughput during user metrics monitoring when the priority of Netdata is increased.
One might expect that simply raising the priority of Netdata would enable low-latency monitoring.
However, we show that increasing the priority alone does not fully resolve the problem.
In this experiment, we configure Netdata to run under the SCHED\_FIFO scheduling policy with a priority of 99.



\textbf{Monitoring Latency. }
% We show the monitoring latency when observing 10 Memcached instances with a 10 ms sampling interval in Fig.~\ref{fig:monitoring-latency-three}.
We show the monitoring latency for 10 Memcached servers with a 10ms sampling interval in Fig.\ref{fig:monitoring-latency-three}.
The CDF in Fig.~\ref{fig:cdf-u-met-pri} illustrates that prioritized Netdata exhibits monitoring latency that lies between {\sysname} and Netdata up to around the 98th percentile.
Around the point where it begins to exceed the latency of standard Netdata, the tail latency starts to increase sharply.
One possible cause of this behavior is that, under the prioritized Netdata configuration, the Memcached API may fail to respond in a timely manner.
Fig.~\ref{fig:elapsed-latency-nonpri} and Fig.~\ref{fig:elapsed-latency-pri} show the relationship between monitoring elapsed time and monitoring latency for non-prioritized Netdata and prioritized Netdata respectively.
In both graphs, we observe latency spikes occurring at one-second intervals.
These spikes correspond to the moments when Netdata’s child process, \texttt{go.d.plugin}, is invoked to collect user metrics.
At these points, the plugin issues API requests to Memcached, and the resulting waiting time contributes to the latency.
Outside of these spike intervals, the prioritized Netdata is able to execute monitoring tasks more promptly, resulting in lower latency.
In contrast, during the spike intervals, the latency becomes higher with prioritization.
This is because prioritizing Netdata increases its CPU scheduling preference, which can in turn delay the execution of the Memcached API itself.
As a result, \texttt{go.d.plugin} must wait longer for the Memcached response, thereby increasing the monitoring latency during these periods.

\textbf{Impact of Monitoring on Host Service Performance. }
Fig.~\ref{fig:throughput-u-met-pri} presents the Memcached throughput at various sampling intervals when monitoring 10 Memcached instances.
For both non-prioritized Netdata and prioritized Netdata, the throughput shows no significant difference across all sampling intervals.
However, in both cases, the Memcached throughput decreases as the sampling interval shortens.
This indicates that, regardless of prioritization, Netdata continues to degrade the performance of the Memcached service under frequent monitoring.
\section{Relatedwork}
\label{sec:related-work}

\textbf{Monitoring system in distributed environments. }
Numerous studies have been dedicated to the design of monitoring systems.
Many of these studies primarily concentrate on aspects such as data analytics~\cite{monalytics, sieve}, bug tracing~\cite{partial-failures, pivot}, and visualization~\cite{health-monitoring}.

Zero~\cite{zero} concentrates on low-overhead and low-latency monitoring using RDMA.
This framework doesn't use the host CPU to collect metrics and packet processing, which leads to almost zero overhead and low latency despite high loads of the monitored server.
In this approach, all monitored servers require RNIC, which is expensive hardware.
Although {\sysname} uses a small portion of the host's CPU, it operates over conventional Ethernet, making it easy to deploy in cloud environments.

Safetimer~\cite{safetimer} is another approach to prevent system failure due to the false detection of failure.
This framework enhances existing timeout detection protocols to tolerate long delays of monitoring messages.
In this approach, however, since it does not aim for low-latency monitoring, it may fail to capture the frequently changing metrics of cloud environments in real time.
{\sysname} can prevent false timeout detection, by achieving low-latency monitoring.

\textbf{eBPF application. }
eBPF was only used for packet filtering~\cite{BSD}, and load balancing~\cite{load-balancer} because of its restricted programming model from the beginning.
These days, eBPF can be used to offload more operations into kernel space which are small yet critical operations.
XRP~\cite{xrp} allows applications to execute user-defined storage functions in the NVMe driver, safely bypassing most of the kernel storage stack to achieve low-overhead file I/O operations.
Electrode~\cite{electrode} accelerates distributed protocols using XDP packet processing.
SynCord~\cite{SynCord} leverages eBPF to inject workload-specific and hardware-aware kernel lock policies.

\textbf{Kernel-bypass packet processing. }
Due to the overhead associated with the monolithic kernel networking stack, many efforts have turned to kernel bypass techniques to achieve lower latency and reduced overhead.
Demikernel~\cite{demikernel}, mTCP~\cite{mtcp}, eRPC~\cite{datacenter-rpcs} attempt to eliminate the kernel from the I/O datapath.
In general, the drawback of granting users direct access to I/O is that applications must actively poll for I/O to achieve high performance.
As a result, cores cannot be shared among processes, leading to substantial under-utilization when I/O is not the primary bottleneck.
{\sysname} leverages eBPF to unclog some of the bottlenecks in the kernel networking stack without completely shifting kernel bypass.
\section{Conclusion}
\label{sec:conclusion}

In cloud environments transitioning to microservices, traditional monitoring methods no longer meet the demands of cloud-native monitoring.
We propose {\sysname}, a lightweight monitoring framework that extends XDP with minimal modifications.
{\sysname} completes monitoring tasks within the SoftIRQ layer, bypassing delays caused by CPU scheduling and the overhead associated with user-to-kernel context switches in traditional approaches.
Despite its lightweight design, {\sysname} maintains the same metric collection capabilities as conventional monitoring systems.
Our experimental results demonstrate that {\sysname} achieves both low-latency monitoring and minimal interference with host services, outperforming traditional monitoring methods.

\bibliographystyle{IEEEtran}
\bibliography{reference}

\end{document}
