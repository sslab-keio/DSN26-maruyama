\section{X-Monitor}
\label{sec:proposal}

% \begin{figure*}[t]
%    \centering
%     \includegraphics[width=0.75\textwidth]{SoCC-2025/figures/illustration/run-to-completion.pdf}
%    \caption{Execution Model Comparison. (1) In traditional monitoring systems, execution may be delayed due to lock contention within the network stack and waiting for CPU scheduling. (2) In contrast, {\sysname} adopts a run-to-completion model: once execution begins, it proceeds without being preempted or blocked by lock contention or scheduling delays, ensuring uninterrupted processing from start to finish.
%    }
%    \label{fig:run-to-completion}
%\end{figure*}

We propose {\sysname}, a lightweight monitoring system for cloud-native services that achieves low-latency monitoring while minimizing interference with the execution of microservices.
{\sysname} leverages eXpress Data Path (XDP), a lightweight packet processing mechanism, and provides the following three key features:

\begin{enumerate}
    \item \textbf{In-Kernel Monitoring:} Monitoring is completed entirely in kernel space, eliminating user–kernel context switches.
    \item \textbf{SoftIRQ-Layer Execution:} Since monitoring is performed at the SoftIRQ layer, there is no need to wait for CPU allocation due to scheduling.
    \item \textbf{Programmable Metric Selection:} Since the monitoring logic is implemented as an XDP program, users can flexibly choose which metrics to collect.
\end{enumerate}

Major cloud providers, including Amazon Web Services (AWS), Microsoft Azure, and Google Cloud Platform, offer Ethernet-based NICs as the default option for most instance types.
% While advanced networking technologies such as Remote Direct Memory Access (RDMA) are available in these environments, they often involve additional costs or configuration overhead.
While advanced networking technologies such as Remote Direct Memory Access (RDMA) are available in these environments, their usage is limited to specific instance types, which are often expensive.
For example, AWS offers Elastic Fabric Adapter (EFA), an RDMA-capable NIC designed for HPC and ML workloads.
EFA is only available on a limited set of instance types, such as c8gn.48xlarge, m6i.metal, and p6-b200.48xlarge~\cite{aws-efa-doc}.
These EFA-enabled instances are generally more expensive, making RDMA less accessible for cost-sensitive deployments.


To demonstrate that {\sysname} can support diverse and practical monitoring use cases, we examine how typical metrics are managed in most software systems.

\textbf{Metric Characteristics. }
Most monitoring metrics are simple counters that are updated in place after initialization.
Since the memory addresses of these counters remain unchanged,
{\sysname} can reliably collect metrics by referencing the memory addresses registered during initialization.
These characteristics are widely observed across both kernel and user metrics---such as per-CPU packet counts, memory usage statistics, and request rates.
As a result, {\sysname} can support a broad range of metrics comparable to traditional monitoring systems, while maintaining a lightweight and efficient design.

\textbf{XDP Overview. }
{\sysname} is implemented using XDP, an eBPF-based framework that enables high-performance packet processing in the kernel space.
As illustrated in Fig.~\ref{fig:overview-xdp}, XDP programs are triggered upon packet arrival and execute at the device driver layer, specifically in the SoftIRQ context, immediately after the hardware interrupt.
This design allows XDP to bypass the conventional Linux networking stack—prior to \texttt{sk\_buff} allocation—thereby achieving low-latency packet processing.
To ensure the safety and reliability of kernel execution, XDP includes a static verification mechanism, known as verifier~\cite{verifier}, which enforces strict constraints on loaded programs.
Verifier checks for invalid memory accesses, unbounded loops, and other unsafe behaviors, allowing only safe and well-formed programs to be loaded into the kernel.

\textbf{Challenges. }
Building {\sysname} atop XDP introduces two main challenges.  
First, the monitoring logic must pass verifier’s checks, which is non-trivial when accessing memory regions used to store metrics.  
Direct access to external memory is typically disallowed by verifier, requiring a mechanism that safely exposes metric memory to the XDP context while satisfying all verification rules.

Second, {\sysname} must ensure that metric collection is strictly non-blocking.
Since XDP runs in the SoftIRQ context, any blocking operations—such as waiting for locks or accessing swapped-out memory—could stall the system and degrade performance.  
To ensure robust system behavior, {\sysname} guarantees that all metric accesses are safe and non-blocking during monitoring execution.